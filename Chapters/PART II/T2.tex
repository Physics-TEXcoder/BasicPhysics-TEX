\section{理想气体状态方程}
我们已经知道,当温度较高、压强趋于零的各种气体的
宏观状态都遵从玻意耳定律、查理定律和盖吕萨克定律。
满足这些性质的气体称为理想气体(ideal~gas)。
显然,理想气体是实际气体在压强很小的情况下的极限,是一个气体模型。

根据玻意耳定律,我们知道理想气体的压强$p$和体积$V$的乘积是由温度
$T$决定的常量,即有
\begin{equation}
    pV=C(T).
\end{equation}
现在我们考察体积始终为$V_0$的定容理想气体,并记$T_0,p_0$和$T,p$
分别为这些气体的标准状态的固定状态和可以变化的任意状态。那么 
根据盖吕萨克定律,我们有
\begin{equation}
    \frac{p}{p_0}=\frac{T}{T_0}=\frac{pV_0}{p_0V_0}=\frac{C(T)}{C(T_0)}
\end{equation}
进一步地
\begin{equation}
    C(T)=\frac{C(T_0)}{T_0}T=\frac{p_0V_0}{T_0}T.
\end{equation}
又由于实验表明,温度为水的冰点273.15K、压强为1atm的情况(这一
情况常被标准状况)下,1mol的任何气体的体系$V_m$都为22.4144升(L)。
那么,我们选取1mol的处于标准状况的任意理想气体作为标准状况,则
对于1mol的任意理想气体,都有
\begin{equation}
    \frac{p_0V_0}{T_0}=8.314510J/(mol\cdot K)
\end{equation}
是一个常量,称之为普适气体常量(universal~gas~constant),
简记为$R$。所以,1mol理想气体的压强$p$,体积$V_m$
和温度$T$之间的关系(数值上)可以表示为
\begin{equation}
    pV_m=RT.
\end{equation}
该关系式为1mol理想气体的状态方程。对于$\nu$mol的理想气体,其体积
为 $V=\nu V_m$,则$\nu$mol理想气体的状态方程为
\begin{equation}
    pV=\nu RT.
\end{equation}
由于一定量气体的摩尔数$\nu$可以由其质量$M$和摩尔质量$\mu$表示为
$\nu=\frac{M}{\mu}$,故质量为$M$的理想气体的状态方程为
\begin{equation}
    pV=\frac{M}{\mu}RT.
\end{equation}
\section{固体和液体的性质}
\subsection{物态简介}
\begin{definition}
    构成物质的分子的聚合状态称为物质的凝聚态,简称物态。
\end{definition}

\begin{theorem}
    常见物质的表观特征,可以粗略地分为三类:
    
    (1)固体:没有流动性,不易被压缩,具有一定的体积和确定的形状,并具有弹性;
    
    (2)气体:具有流动性,容易被压缩,本身没有确定的体积和确定的形状,没有 
    明显的附着性,没有弹性;
    
    (3)液体:具有流动性,但不易被压缩,有确定的体积,但没有一定的形状,有 
    明显的附着性。
\end{theorem}

通常称为凝聚态物体的固体和液体,分子数密度比气体分子数密度大$3\sim 4$个 
数量级,固、液体内分子间的平均距离比气体小一个数量级。

分子密集系统(固、液体)内分子之间的相互作用力比气体内分子之间的作用力
要强得多。

固体内分子间平均距离($10^{-10}m$数量级)比液体更小些,固体分子只能分布在
一定位置附近做热运动,这使得固体在宏观上具有一定的形状和体积。

物质的凝聚态的微观解释在第一章便有提及,此处不再重复。
\subsection{固体的性质简介}
\subsubsection*{1、固体的分类}
\begin{theorem}
    固体分为两大类:晶体和非晶体。晶体又分为单晶体与多晶体。
    
    单晶体是指样品中所含分子(原子或离子)在三维空间中呈规则、周期排列的
    一种固体状态;多晶体实际上是由很多外形不规则的小单晶体晶粒混乱组成的,
    晶粒尺寸一般为$10^{-4}\sim 10^{-3}cm$,最大的可达$10^{-2}cm$。
    
    非晶体是指结构无序或者近程有序而长程无序的物质,组成物质的分子(或原子、离子)
    不呈空间有规则周期性排列的固体,它没有一定规则的外形。
\end{theorem}
\subsubsection*{2、单晶体物体的宏观特性}
\begin{theorem}
    (1)晶面角恒定不变(晶面角守恒定律)
    
    从外形上看单晶体都是由光滑平面所围成的凸多面体,这些光滑的
    平面叫做晶面。同一种晶体由于生长过程中所处的外界环境条件不同,晶体 
    物体的大小和形状可能不同,但晶体相应晶面之间的夹角却是保持恒定不变的。
    
    (2)各向异性
    
    单晶体内各个不同方向的力学性质、热学性质、电学性质、光学性质等
    一般各不相同,称为各向异性性质。
    
    (3)晶体具有确定的熔点
    
    晶体物质在一定压强下,总是被加热到一定温度才开始熔化,并且在熔化过程中晶体
    温度保持不变,直至晶体完全溶解成液体后温度才会继续升高。晶体开始熔化时的
    温度称为晶体的熔点。
\end{theorem}
\subsubsection*{3、晶体与非晶体的差异}
\begin{theorem}
    具有确定的熔点是晶体 (包括单晶体和多晶体)与非晶体的一个基本区别。
\end{theorem}

像玻璃、沥青、石蜡等这类非晶体被加热时,随着温度的升高先由硬变软,然后逐渐
由稠变稀,然后变成为可流动的液体。在这个过程中温度一直在升高,不存在确定的熔点。

正是由于非晶体没有确定的熔点,所以非晶体也被视为是温度太低粘滞系数过大以至
于不能流动的液体,从这种意义上说,只有晶体才算是固体。

晶体和非晶体的差异来自于它们不同的微观结构。晶体在微观上具有周期性空间点阵
结构和高度的对称性,以及微观结构的长程有序和短程有序性;而非晶体粒子的微观
结构具有空间排列的短程有序 (相邻原子或分子之间的相关性)和长程无序的基本特征。
\subsubsection*{4、晶格与结构}
晶格是延伸到整个空间的几何点的周期性重复排列。

在晶体结构中,一个格点代表一个原子或原子团。原子团可能是相同原子也可能是不同
原子。这个原子或原子团叫做基元。完美晶体或者理想单晶,其晶体结构在整个空间是
无限延伸的。
\subsubsection*{5、晶体结合力}
晶体不被热运动所拆散,而以一定规则的有序结构结合成一个整体,是因为晶体中各
原子之间存在着结合力。结合力是决定晶体性质的一个主要因素。

晶体结合力也称为化学键,有共价键、离子键、范德瓦尔斯键、金属键四种类型,
另外, 还有一种处于共价键与离子键之间的结合形式——氢键。
\subsection{液体的性质简介}
液体是一种常见的物质存在形态,其性质介于固体和气体两者之间。液体具有像
气体那样的流动性和没有一定形状的性质。

从实验观察固体熔解和液体结晶过程中发现,大多数物体在固液态转化前后,体积 
只有$10\%$左右的变化,因而分子间的平均距离只改变$3\%$左右。液体同固体一样
也是分子密集聚集系统。

液体分子间的平均距离要比通常条件下气体分子间平均距离小一个数量级。
液体分子间的作用力要比气体分子间的作用力强很多。

在这样强的相互作用力下,使得液体分子不会像气体分子那样总是分散到
不论多大的整个空间,而是使得液体分子紧密地聚集在一定的空间从而
使液体具有一定的体积。

通过X射线衍射研究熔解和结晶过程中液体的微观结构时发现:液体内的分子只在
微小区域(只有几个分子间距范围)内作有规则排列的分布,而且这些微小区域的分子
排列方位完全混乱无序,与非晶体内分子的分布相似。

不过液体内分子有规则排列的微小区域不是固定不变的,微小区域的大小、边界不断
变化,一些分子规则排列微小区域不时在瓦解,而另一些新的分子规则排列微小
区域又不时形成。

液体内分子的分布是不断变化着的短程有序分布,因而除液晶外,液体和非晶态
固体一样呈现出各向同性。

目前对分子热运动占主导地位的气体和分子力占主导地位的固体的理论研究比
较成熟,而对介于这两个极端情况之间的液体的理论研究还很不成熟。
研究液体的方法是从两头逼近的方法:

(1)当液体温度增高接近其沸点时:液体分子热运动比较激烈,其性质与实际气
体靠近,从而更多地把液体设想为稠密的实际气体去分析液体的性质。譬如,在气
液相变中用纯粹是实际气体的范德瓦尔斯方程去解释液体和气体相变行为。

(2)当液体的温度趋向凝结时,又常用频临瓦解的固体去想象液体,突出液体分子
间的关联。我们常采用X射线、中子束、电子束的衍射等研究固体微观结构的
常规方法去研究液体的微观结构,用濒临瓦解的固体图像去解释液体热容、热传导 
等现象。液体的微观结构和非晶态固体类同。

实验表明,液体的导热系数很低,热容与固体接近。扩散系数比固体稍大,但比
气体小很多。同为流体,液体比气体的黏性大,且随温度的升高而降低。液体有着 
非常重要的表面现象,如表面活性、弯曲液面的附加压强、毛细现象等。
