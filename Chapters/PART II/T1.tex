\section{物质结构的基本图像}
\subsection{物质分子处于不停顿的无规则运动状态}
\begin{theorem}
    物质分子都在不停顿地作各向同性的无规则运动。
\end{theorem}

定量地讲,分子无规则
运动的特征是:在其体坐标系(或质心系)中,分子的质心动量为零。分子的 
这种无规则、随机运动又被称为分子的热运动(thermal~motion)。
我们在讨论物质分子的热运动时,应该将之同整体运动区分开来,并将
整体运动扣除掉。

分子热运动的典型表现是布朗运动(Brownian~motion),也就是微小颗粒的
无规则运动。另外,作无规则运动的微小颗粒被称为布朗粒子。\begin{theorem}
    布朗运动
    是由于微小颗粒受到周围分子碰撞的不平衡而引起的一种起伏运动,布朗 
    粒子的无规则跳动与负载布朗粒子的介质的无规则运动互为表里。
\end{theorem}

\subsection{分子之间存在相互作用}
\begin{theorem}
    分子之间存在相互作用力,通常称这种相互作用力为分子力(intermolecular
    ~force)。分子力由吸引力和排斥力两部分构成。
\end{theorem}

根据实验,我们得知分子力在分子相距较远时表现为吸引力,在分子相距
很近时表现为排斥力。这相互作用力是一个有心力,因而它是保守的,
我们可以定义出分子间的相互作用势能$\phi(r)$,这是一个负值,
其绝对值$E_B$称为势阱。

当分子间的间距小于平衡距离$r_0$,尤其是小于有效直径$d$时,
分子之间的排斥力很强,此时分子无法聚合。当分子之间间距远大于平衡距离$r_0$时,
分子间的作用力很弱,此时分子成为自由粒子。只有当分子间距$r\thickapprox r_0$,
即处于势阱中时,才形成间距$r\thickapprox r_0$的束缚状态,从而形成稳定的物质
凝聚态。由于分子除与其他分子之间有相互作用外,还具有无规则热运动,即有动能
$\frac{1}{2}mv^2$,分子的平均动能为
\begin{equation}
    \overline{\varepsilon_k}=\frac{1}{2}m\overline{v^2}.
\end{equation}
那么,在$\overline{\varepsilon_k}\ll  E_B$的情况下,所有分子
都被束缚在势阱宽度所限的区间内运动,从而形成稳定的束缚态,即形成
固态。在$\overline{\varepsilon_k}\gg  E_B$的情况下,物质分子 
的平均动能远大于其间的势阱深度,相互作用对分子运动状态的影响很小,
从而所有分子将尽可能地均匀地充满其能占据的空间,也就是形成气态。
在$\overline{\varepsilon_k}\thickapprox  E_B$地情况下,分子的
动能与分子间的势阱深度相当,分子可以不完全受制于其他分子的束缚,
但不会偏离太远,总的效果是形成介于固态和气态之间的液态。总之,
物质形态与分子间相互作用势能的势阱深度密切相关,如果势阱深度$E_B$
远大于分子的热运动平均动能,则物质呈固态;如果势阱深度$E_B$约等于
分子热运动的平均动能,则物体呈液态;如果势阱深度$E_B$远小于
分子热运动的平均动能,则物质呈气态。
\section{温度与温标}
\subsection{温度的概念}
我们之前已经提到,为标记系统的冷热程度,我们引入一个热学中特有的
物理量。该标记系统冷热程度的物理量称为温度。显然,温度是宏观量。
深入的研究表明,本质上,温度是组成系统的大量微观粒子的无规则运动
剧烈程度的表现和度量。
\subsection{温度相同的判定原则——热力学第零定律}
在热物理学中,人们把由导热壁连接而实现的接触称为热接触。导热壁是
两热力学系统之间位置固定但可以使其两边的系统的状态相互影响的隔板。
与之相对的,两热力学系统之间的位置固定且使两边的系统的状态互不
影响的隔板称为绝热壁。

经验表明,冷热程度不同的两个物体通过热接触,经过一段时间后,
可以达到相同的冷热程度,即达到相同的温度,此时宏观性质不再发生变化,
也就是说两物体达到了热平衡。热平衡和冷热程度相同是共同出现的,因而
我们可以说,当两个系统达到热平衡时,其温度是相同的,热平衡可以作为
温度相同的判据。下面,我们给出判定热力学系统是否达到热平衡的定律。

\begin{theorem}
    实验结果表明,在不受外界影响的条件下,如果两个热力学系统中的每一个
    都与第三个热力学系统处于热平衡,则它们彼此也必定处于热平衡,该规律
    称为热平衡定律,也被称为热力学第零定律(zeroth~law~of~Thermodynamics)。
\end{theorem}
我们可以利用此定律来判定不同的热力学系统间是否达到热平衡,进而
判定其温度是否相同。

\subsection{温度高低的数值标定——温标}
在前面我们已经知道,人们可以使具有确定初始状态的第三个物体与其他多个
物体分别进行热接触来判定这些物体是否处于热平衡状态,即判定它们的
温度是否相同。若不同,第三个物体会在与它们的热接触后表现出不同的变化,
比较第三个物体与它们分别达到热平衡时状态的变化,可以判定不同物体
温度的高低(同理,这种方式也可以判定同一物体处于不同状态时的温度高低)。

显然,只定性地判定物体的冷热程度在物理研究中是不足的,我们希望能够
有方法可以定量地标定出物体的冷热程度,也就是温度的具体数值。

\begin{definition}
    如果我们选中一种物质,在给定的冷热程度下它的某物理量总是确定的
    (即该物理量与温度之间仅有单值函数关系),并且随着温度的变化
    该物理量可以发生明显的变化(即该物理量是对温度敏感的),那么我们就可以
    将该物理量的不同状态标定为不同的数值,那么每个确定的温度都会对应
    唯一的该物理量的一种状态,进而对应着一个确定的数值,这样就可以
    实现温度的数值标定。
    
    在热力学中,这种选定的物质被称为测温物质(thermometric~substance),
    选定用来标定温度的物理量称为测温属性(thermometric~property),
    测温属性与温度之间的函数关系被称为测温曲线,与温度一一对应的测温
    属性的状态被称为标准点(calibration~point),温度的数值表示法被称为
    温标(temperature~scale)。从上面的叙述中,我们不难知道,测温物质、
    测温属性和固定标准点是定量准确地标定温度的基本要素,它们常被称为
    温标三要素。值得注意的是,温标三要素中的测温属性不仅包括选定的
    物理量,还包括其测温曲线;固定标准点还包括指定标准点的温度值。
\end{definition}

在实际生产、生活及科学研究中,使用的温标主要有经验温标(empirical~
temperature~scale)、理想气体温标(ideal~gas~temperature~scale)、
热力学温标(thermodynamic~scale~of~temperature)及国际实用温标
(international~temperature~scale)。

\subsubsection{(i)经验温标}

目前采用的经验温标主要有华氏温标和摄氏温标两种。

华氏温标是德国物理学家华伦海特(G.D.Fahrenheit)与1714年利用水银
在玻璃管内的体积变化而建立的温标。其中,规定氯化氨与水及冰的混合物
的熔点为0度(相当于当地冬天的最低温度),冰与水的混合物的温度为32度,
将在0度与32度之间的一定量的水银的体积(或长度)变化量等分为32格(即 
人为地使测温属性与温度间成线性关系),一格所对应的变化量就代表一
华氏度,常常记为$^{\circ}F$。

摄氏温标是瑞典天文学家摄尔修斯(A.Celsius)与1742年以水银为测温物质
、细玻璃管种水银的体积为测温属性、测温曲线也为线性函数关系而制定的
温标。其与华氏温标的不同在于,规定纯冰与纯水在一个标准大气压下达到
平衡时的温度(常称为水的冰点温度)为0度,纯水与水蒸气在蒸汽压等于一个标
准大气压下达到平衡的温度(常称为水的汽点温度)为100度。
摄氏温标的单位记作$^{\circ}C$。由于在两固定点之间水银的体积
(长度)随温度线性变化,那我们可以记温度$t$与水银柱长度$X$的关系为
\begin{equation}
    t(X)=t_0+kX.
\end{equation}
再记$t=0^{\circ}C$时水银柱的长度为$X_i$,$t=100^{\circ}C$时水银柱 
的长度为$X_s$,则有
\begin{equation}
    t(X)=100\frac{X-X_i}{X_s-X_i}.
\end{equation}
于是从水银柱的高度$X$可以直接读得温度$t$的数值。据此做成的用来测定
其他物体的温度的标准装置即摄氏温度计(Celsius~thermomenter)。

另外,华氏温标与摄氏温标的关系为
\begin{equation}
    t/^{\circ}F=32+\frac{9}{5}t/^{\circ}C.
\end{equation}

\subsubsection{(ii)理想气体温标}

在经验温标下,测温物质和测温属性可能千差万别。利用由不同测温物质
及相应的测温属性做成的温度计测量某一处于确定状态的系统的温度时会
得到不同的结果。因此,为了使温度测量得到确定、一致的结果,必须建
立统一的温标作为标准。实验结果表明,不同的气体温度计的测量结果比
较接近。于是,为了建立统一的温标或许可以选用气体作为测温物质。

对于一定量的稀薄气体,实验表明,在体积固定的条件下,压强$p$与摄氏
温标下的温度$t$有关系
\begin{equation}
    p=p_0(1+\alpha_pt).
\end{equation}
其中,$p_0$是摄氏温度为0时的压强。
这就是著名的盖吕萨克定律(L.J.Gay-Lussac于1809年提出。亦有人称之为
阿蒙东定律,因为阿蒙东(G.Amontons)于1669年注意到其部分迹象)。并且,
当$p_0\rightarrow 0$时,任何气体的$\alpha_p$都趋于同一个常量。
即在$p_0\rightarrow 0$时,我们有
\begin{equation}
    \lim \limits_{p_0\rightarrow 0}\alpha_p=\alpha_0.
\end{equation}
因此,对于任意满足$p_0\rightarrow 0$的气体都有
\begin{equation}
    p=p_0(1+\alpha_0t).
\end{equation}
成立。显然$\alpha_0$的量纲为温度的倒数,故我们可以令
\begin{equation}
    T_0=\frac{1}{\alpha_0}.
\end{equation}
这个常数$T_0$很可能是与温度相关联的。我们用$\alpha_0=
\frac{1}{T_0}$作代换,是有利于我们发现公式的含义的。
于是我们可以将(5.7)式写成
\begin{equation}
    p=p_0\frac{T_0+t}{T_0}.
\end{equation}
即 
\begin{equation}
    \frac{p}{T_0+t}=\frac{p_0}{T_0}.
\end{equation}
受到上式形式以及量纲的启发,我们如果能够提出一个新的温标定义,使得
上式分母上的两个值分别代表两个状态的温度,那么我们就会得到一个
漂亮的结果,即定容稀薄气体的压强与温度的比值总是为定值。
因此我们定义
\begin{equation}
    T=T_0+t
\end{equation}
为新的温标。在此定义下,(5.10)左边分母是代表该状态下的新温度,
右边分母$T_0$也是代表压强为$p_0$时的新温度,可见此定义是极好的。

经过对多种稀薄气体的测量,我们得到了
\begin{equation}
    T_0=273.15.
\end{equation}
即新定义下的温标与摄氏温标的关系为
\begin{equation}
    T=273.15+t.
\end{equation}
现在我们看似定义出了完整的新温标,但是,这个新温标数值的标定是基于
摄氏温标的。而我们希望不依赖其他温标,仅仅通过测出气体的某种测温
属性(在这里指压强$p$)就得到温度值。

若记某个固定状态的温度和压强分别为$\widetilde{T_0},\widetilde{p_0}$,
结合(5.10)(5.11)式,得到 
\begin{equation}
    T=p\frac{\widetilde{T_0}}{\widetilde{p_0}}.
\end{equation}
因而我们只需要找出某个固定的标准状态来定出标准点,就可以
使得这个新温标具有自己的标定方式(即用压强标定)。

由于二相共存的状态的压强和温度具有不确定性,我们不好找到某一个特定的
二相共存的状态的压强和温度(这一点在之后学习三相图时会更加容易理解),
故二相共存的状态不适合用作标准状态。而我们之后会知道,三相点的状态是
固定的,只要物质处于三相点,仅仅会对应一个温度和压强。
因此,出于精确、可重现的考虑,我们需要找到某物质的三相点来
敲定这一比值。

我们实际上选择了所谓$H_2O$的三相点
(triple~point,即冰、水和水蒸气三相共存,并达到平衡的状态)
作为标准状态。当水处于三相点时,其摄氏温度为$t=0.01^{\circ}C$,
其对应的新温标为$\widetilde{T_0}=273.16$。接着测出此时的
压强$\widetilde{p_0}$,代入到(5.14)式,即可得到
\begin{equation}
    T=T_0\frac{p}{\widetilde{p_0}}=273.16\frac{p}{\widetilde{p_0}}.
\end{equation}
我们记定容气体的温度为$T_V$,因而上式可以更精确地写作
\begin{equation}
    T_V(p)=273.16\lim \limits_{p_0\rightarrow 0}\frac{p}{\widetilde{p_0}}.
\end{equation}
这样,我们在规定了一个小压强的固定点的情况下就建立了不依赖于具体化学
组分的定容气体温标。据此可以做成定容气体温度计。
事实上,只要选定好标准状态进行标定,得到的新温标仍然满足
定义中摄氏温标与新温标的换算关系
\begin{equation}
    T=273.15+t.
\end{equation}

实验还表明,在压强保持为$p_0$不变的情况下,稀薄气体的体积与以摄氏温标
表示的温度$t$之间有关系
\begin{equation}
    V=V_0(1+\alpha_Vt).
\end{equation}
并且当$p_0\rightarrow 0$时,对任何气体$\alpha_V$都趋于同一
常量,即有 
\begin{equation}
    \lim \limits_{p_0\rightarrow 0}\alpha_V=\alpha_0=\frac{1}{T_0}.
\end{equation}
定义$T=T_0+t$,我们有 
\begin{equation}
    V=\frac{V_0}{T_0}T.
\end{equation}
这就是著名的查理定律(J.A.C.Charles于1787年提出),亦称为查理-盖吕萨克
定律(盖吕萨克确认查理的结果,并于1802年1月公布,也有人称之为查理-道尔顿-
盖吕萨克定律,因为道尔顿(J.Dolton)曾于1801年10月发文说明有此规律)。显然,
采用与建立定容气体温标相同的方案规定固定点及相应的温度(这样规定
可以使得对于某一确定状态,定压气体温标与定容气体温标是相等的)
,我们可以建立定压气体温标
\begin{equation}
    T_p(V)=273.16\lim \limits_{p_0\rightarrow 0}\frac{V}{V_0}.
\end{equation}
据此也可以做成定压气体温度计。由于定压气体温度计的结构比定容气体温度计的
结构复杂得多,并且实用时操作也麻烦得多,实际中较少使用。显然地,
此时的定压气体温标与摄氏温标之间的换算关系也是
\begin{equation}
    T_p=T_V=273.15+t.
\end{equation}

实验上还发现,压强趋于0的气体还遵从玻意耳定律,即当温度保持不变时,
一定量的气体的压强和体积的乘积为一个常量。严格遵从查理定律、盖吕萨克 
定律及玻意耳定律的气体称为理想气体。由上述讨论可知,当压强趋于零时,
实际气体很好地近似为理想气体,于是上述从定容和定压两种角度建立的气体温标
统称为理想气体温标,通常记为T,单位为开尔文(Kelvin),简记为K。

由(5.13)式和(5.17)式,我们可以看到,理想气体温标与摄氏温标间的
关系满足
\begin{equation}
    T/K=273.15+t/^{\circ}C.
\end{equation}
实验还表明,以理想气体温标为基础制造的理想气体温度计确实与所用的
工作物质的化学组分无关,即不依赖于具体的气体。

\subsubsection{(iii)热力学温标}

热力学理论表明,在热力学第二定律基础上可以建立不依赖于任何物质的
具体测温属性的温标(参见chapter5)。这种不依赖于任何测温物质的具体
测温属性的温标称为热力学温标或绝对温标(absolute~temperature~scale),
由之确定或标记的温度称为热力学温度(Thermodynamic~temperature)或 
绝对温度(absolute~temperature)。

在热力学温标中,规定热力学温度是基本的物理量,其单位为开尔文(简记为K),
1K定义为水的三相点的热力学温度的$\frac{1}{273.16}$。这就是说,水的三相点
的温度定义为273.16K。

可以证明,在理想气体温标适用的温度范围内,理想气体温标 
是热力学温标的具体实现方式。

\subsubsection{(iv)国际实用温标}

由于以理想气体温标为基础的气体温度计的结构和使用都很复杂,
一般只在少数国家的核心计量机构才建立这类装置,因此利用气体
温度计直接确定热力学温度很繁复,且不能在国际上推广和普遍采用。
为解决这些问题,国际上通过约定一系列物质的温度的固定点、特殊温区
内作为标准测量用的内插仪器及其测温属性,形成了国际实用温标,
简记为ITS。国际实用温标于1927年制定第一版, 其后曾于1956年、1960
年、1968年、1990年进行修订。现在国际上采用1990年的国际温标(ITS-90)。

ITS-90选取了从平衡氢三相点(13.8033K)到铜凝固点(1357.77K)间16个
固定的平衡点温度,用这些固定点将温度分为若干区域,每个温度区又规定
一些测准仪器(如铂电阻温度计、铂铑热电偶温度计等)去测量在同一温区
的不同温度范围内给出不同的测温关系式。这种方法保证了国际间的温度标准
在相当精确的范围内一致,并尽可能接近热力学温标。

\section{物体的内能}
\subsection{分子的动能}
前面我们已经提及,组成物体的分子具有热运动的动能。物体里分子运动的速率是
不同的,有的大,有的小,因此各个分子的动能
并不相同、在热现象的研究中,我们所关心的不是物体里每个分子的动能,而
是所有分子的动能的平均值。这个平均值叫做分子热运动的平均动能。

可以证明,对于理想气体来说,分子热运动的平均动能是正比于热力学温度的。
可以说,温度是物体分子热运动的平均动能的标志。
\subsection{物体的内能}
\begin{definition}
    物体中所有分子的热运动的动能和分子势能的总和,叫做物体的内能。
    一切物体都是由不停地做无规则热运动并且相互作用着的分子组成的,因此任何
    物体都具有内能。
\end{definition}

由于分子热运动的平均动能跟温度有关系,分子势能跟体积有关系,因此
物体的内能跟物体的温度和体积有关系。温度升高时,分子的动能增加,因而
物体的内能增加。体积变化时,分子势能发生变化,因而物体的内能发生变化。

同样地,理想气体的内能正比于其热力学温度,并且仅与温度有关。
\section{热力学第一定律}
\begin{theorem}
    热力学系统与外界之间的相互作用可以分为力学的和热学的。在这些作用下,
    一方面,系统的状态会发生变化,从而作为态函数的内能会随之改变。另一
    方面,伴随有作功和传热两种形式的能量传递。根据能量守恒定律,作功、
    传热和内能改变这三种形式的能量的总和应保持守恒。于是,当热力学系
    统的状态发生变化时,可以通过作功和传热等方式改变系统的内能,内能
    的增量等于外界对系统所作的功与外界传递给系统的热量之和,此即能量
    守恒定律在涉及热现象宏观过程中的具体表述,也就是著名的热力学第一
    定律 (frst law of thermodynamics)。记热力学过程中外界对系统所作的
    功为$W$,传递给系统的热量为$Q$,系统处于初态$i$ 时的内能为$U_i$,达
    到末态 $f$ 时的内能为$U_f$,内能的增量$U_f-U_i$ 记为 $\Delta U$,则
    热力学第一定律可以用数学形式表示为
    \begin{equation}
        \Delta U=U_{f}-U_{i}=W+Q.
    \end{equation}
\end{theorem}
上述讨论表明,在一个热力学过程中,外界对系统所作的功 $W$ 和传递
的热量 $Q$ 都是代数量,都可正可负、根据前述讨论,将这些量的符号
约定如下:$W>0$ 表示外界对系统作正功,$W<0$ 表示外界对系统作负
功,实质上也就是系统对外界作正功; $Q>0$ 表示外界传递给系统热
量,$Q<0$ 表示外界传递给系统负热量,也就是系统向外界释放热量。
通常,为区分系统对外界作功 (或放热) 和外界对系统作功 (或传热), 
分别用$W^{\prime},Q^{\prime}$表示系统对外界所作的功和系统传递
给外界的热量。对于内能的增量$\Delta U,\Delta U>0$ 表示系统内
能增加 $(U_f>U_i),\Delta U<0$ 表示系统内能减少 $(U_f<U_i)$。

因为功、热量和内能都是能量,则在利用式$(5.24)$进行计算时,就
应都以能量单位“焦耳”(J)为单位。

考虑到热力学第一定律是能量守恒定律在涉及热现象的过程中的具体表述,
热力学第一定律还可以表述为第一类永动机是不可能造成的。所谓第一类
永动机就是不需要消耗任何形式的能量和动力而能对外作功的机械。
显然这与能量守恒定律矛盾,所以是不可能的。事实上,热力学第一定律
的“第一类永动机是不可能造成的”表述是亥姆霍兹 (F. V. Helmholtz) 的
原始表述。