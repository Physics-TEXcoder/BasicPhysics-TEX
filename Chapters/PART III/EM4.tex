\section{交流电}
\subsection{交流电的产生}
我们以简化的交流发电机为例,记发电机线圈处于匀强磁场,并且切割磁感线
的边长为$l$,旋转角速度为$\omega$,在旋转过程中,交流发电机线圈的磁通量为
\begin{equation}
    \phi=Bl\frac{2v}{\omega }\cos\omega t
\end{equation}
因此 
\begin{equation}
    \mathcal{E}=-\frac{d\phi}{dt}=
    2Blv\sin\omega t
\end{equation}
这就是交流电的产生原理。
\subsection{交流电的有效值}
所谓交流电的有效值,即电流或者电压的二次方对于时间的平均值的二次根值。
也就是说
\begin{equation}
    U_{\text{有效}}=\sqrt{\frac{\sum U^2\Delta t}{T}},~~
    I_{\text{有效}}=\sqrt{\frac{\sum I^2\Delta t}{T}}
\end{equation}
从定义可以看出,有效值的意义是,可以用来求等效的热功,比如 
\begin{equation}
    W=\frac{\sum U^2\Delta t}{R}=\frac{U_{\text{有效}}^2T}{R}
    =\sum I^2R\Delta t=I_{\text{有效}}^2RT
\end{equation}

求算交流电的有效值需要用到积分,因此我们不再写出计算过程,而是直接给出
结果:
\begin{equation}
    U_{\text{有效}}=\frac{U_{\text{最大}}}{\sqrt{2}},~~
    I_{\text{有效}}=\frac{I_{\text{最大}}}{\sqrt{2}}
\end{equation}
显然对于纯电阻电流,电压和电流的有效值之间有关系 
\begin{equation}
    U_{\text{有效}}=I_{\text{有效}}R
\end{equation}
因此 
\begin{equation}
    U_{\text{有效}}I_{\text{有效}}=\frac{U_{\text{有效}}^2}{R}
    =I_{\text{有效}}^2R
\end{equation}
也就是说,$U_{\text{有效}}I_{\text{有效}}$也可以用来表征交流电的有效功率。
\section{电磁波}
电磁波在中学阶段没什么好说的。麦克斯韦总结出四个麦克斯韦方程后,将其联立可以
得到磁场或者电场关于时间和位置的二阶偏微分方程,求解它就会是一个周期性传播的
“电磁波”,但这远远超出了中学物理的范围,因此我们不再介绍。这一部分,对于中学生
来说,只需要读一读课本即可。