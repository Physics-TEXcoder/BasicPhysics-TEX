\section{电荷}
大家知道,用丝绢或者毛皮摩擦过的玻璃、塑料、硬橡胶等都能吸引
轻小物体,这表明它们在摩擦后进入一种特别的状态。我们把处于这种
状态的物体称为带电体,并说它们带有电荷。大量实验表明,自然界的
电荷只有两种,一种与丝绸摩擦过的玻璃棒的电荷相同,叫做正电荷;
另一种与毛皮摩擦过的橡胶棒的电荷相同,叫负电荷[正负电荷的称谓
是由富兰克林(Franklin)提出的]。同种电荷间有斥力,异种电荷
间有吸力。

顺便要提的是,我们称它们为正电荷和负电荷,只是因为历史上一直
这么叫着,名字只是历史的巧合而已。电子的电性本质上并没有什么
正负之分。它不像负数,两个负数一旦进行乘法运算,符号就会改变,
但是两个负电荷无法相互作用成为一个正电荷,这与正负数没有可比性。

允许电荷流动的物体叫导体,导体的电阻与温度正相关;不允许电荷流动
的物体叫绝缘体或电介质(绝缘介质),它们的电阻与温度负相关。
此外,还有一种导电性介于导体与绝缘体之间而且电性质非常特殊的
材料,称为半导体,半导体的电阻通常情况下与温度负相关。

利用物质的微观结构可对物体的带电以及不同物体具有不同的导电性作出
解释。物体由微观粒子(主要是质子、中子和电子)构成。电子带负电荷,
质子带有与电子电荷等值反号的正电荷。当物体由于某种原因获得(失去)
某些电子时便处于带电状态。金属之所以导电,是因为内部存在许多自由
电子,它们可以摆脱原子核的束缚而自由地在金属内部运动。电解液之所
以导电,是因为内部存在许多能做宏观运动的正、负离子。反之,在绝缘
体内部,由于电子受到原子核的束缚,基本上没有自由电子,因此呈绝缘
性质。

大量实验证明,在一个与外界没有电荷交换的系统内(最大的系统就是整
个宇宙),正负电荷的代数和在任何物理过程中始终保持不变,这称为电
荷守恒定律,它反映了电荷的一种重要特性,是物理学的重要规律之一。

电荷的另一重要特性是它的量子化,即任何带电体的电荷都只能是某一
基本单位的整数倍。这个基本单位就是质子所带的电荷,称为元电荷,
通常记为$e$。
\section{库仑定律}
\subsection{库仑定律}
如果带电体的线度比带电体之间的距离小得多,那么带电体的大小、形状
及电荷的分布情况对静电力的影响就会小得多,我们可以忽略这些因素的
影响,认为带电体是一个点,这样问题就会大大简化。满足这个条件的带
电体称为点带电体或点电荷。

前面我们已经知道,电荷之间存在静电力的作用。法国科学家库仑
(Coulomb)注意到,电荷之间的静电力与万有引力有许多相似之处,
因而他大胆地猜想静电力的规律与万有引力定律有类似的形式,
接着便进行了一些实验对相对于惯性系静止的两个点电荷之间的
静电力服从的规律进行探究。

首先,库仑在不改变两点带电体的带电量的情况下,仅仅改变了两点带电体
之间的距离,并分别测量出不同距离下的静电力,得到了如下的结论:

(i)两点电荷之间的静电力大小相等、方向相反,并且其方向沿着它们的
连线;静电力在同号电荷间表现为斥力,异号电荷间表现为吸引力。

(ii)静电力的大小与距离$r$的平方成反比。

仿照着万有引力定律,库仑猜想静电力的大小还与各自的电荷量$q_1$及 
$q_2$成正比,但在库仑时代,电荷还只有定性的概念,根据已有的概念,
仅仅只能谈到一个物体是否带电,却无从确定它带电的数量。为了找到
静电力与电荷的关系,库仑使用了一个巧妙但不够严格的方法。他从
对称性的考虑断定,令一个带点金属球与半径、材料完全相同的另一
不带电金属球接触后分开,每球的电荷应该是原带电球的电荷量的一半。
他用这个方法证实了:

(iii)静电力的大小与两点电荷各自的电荷量$q_1$及 $q_2$成正比。

结论(ii)(iii)实际可以表示为如下的数学形式: 
\begin{equation}
    F=k\frac{q_1q_2}{r^2},
\end{equation}
其中$k$是比例常数,依赖于各物理量单位的选取。上述规律被称为
库仑定律。
\subsection{高斯对电荷的定量探索及高斯制}
我们前面已经提到,电荷还未有定量的定义。电荷后面作为一个
物理量的严格定义主要是由高斯(Gauss)作出的,定义过程如下。

设有$A$、$B$、$C$三个点电荷。先令$A$与$C$间的距离为$r$,用 
扭秤测出它们的静电力$F_{AC}$。再令$B$与$C$间有相同的距离,
测出它们的静电力$F_{BC}$。记下这两个力的比值$F_{AC}/F_{BC}$。
用其他点电荷$D$、$E$代替$C$重复以上实验,发现
\begin{equation}
    \frac{F_{AC}}{F_{BC}}=\frac{F_{AD}}{F_{BD}}
    =\frac{F_{AE}}{F_{BE}}=\cdots ,
\end{equation}
表明这个比值只取决于点电荷$A,B$而与第三个点电荷无关。改变
距离$r$重复以上实验,发现(7.2)仍成立。可见,比值$F_{AC}/F_{BC}$
反应$A$与$B$本身的带电性质。

再考虑到在库仑简陋的电荷“定义”下,静电力与两点电荷各自的电荷量
$q_1$及 $q_2$成正比,故我们可以定义比值$F_{AC}/F_{BC}$为$A,B$的
电荷之比。以$q_A,q_B$分别代表$A,B$的电荷(暂时还没有定义),我们 
定义 
\begin{equation}
    \frac{q_B}{q_A}=\frac{F_{AC}}{F_{BC}}.
\end{equation}
任意指定$B$的电荷为一个单位(即指定$q_B=1$),便有
\begin{equation}
    q_A=\frac{F_{AC}}{F_{BC}}.
\end{equation}
这就是电荷的高斯定义,它提供了一种测量电荷的方法:为测某个
带电体的电荷,只需令它为$A$并与选作单位的点电荷$B$及任一
点电荷$C$进行实验,测出$F_{AC}/F_{BC}$,便得$A$的电荷$q_A$。

上面的定义告诉我们,单位电荷$q_B$的选择只不过会使得不同电荷量
的数值被等比例地缩小或放大一定的倍数。为了简便起见,我们
以力学中的厘米$\cdot $克$\cdot $秒制(CGS制)下库仑公式中比例
系数$k=1$来确定$q_B$,即CGS制下库仑定律可以被写为
\begin{equation}
    F=\frac{q_1q_2}{r^2}
\end{equation}
的形式。依此我们确定出选定为一个单位的电荷$q_B$,从而有了电荷量
的定义。这种定义下的电荷的单位为静库(statcoulomb,简记为SC)。
在此定义下,当两个带电量为1SC的点电荷相距1cm时,产生的静电力就为
1dyn(达因,为CGS制下的力学单位)。
\subsection{国际单位制下电荷量的定义}
国际单位制(SI制)的力学和电磁学部分被称为MKSA制。该制以长度、质量
、时间及电流为基本量,以m(米)、kg(千克)、s(秒)及A(安培)为基本单位。
从这里我们其实可以看出,电流的单位安培实际上是先于电荷的单位定义
出来的。事实上,当我们在真空中的截面积可忽略的两根相距1米的平行
而无限长的圆直导线内通以等量恒定电流,使得导线间相互作用力在1米长度
上为$2\times 10^7N$时,每根导线中的电流就为1安培。而若导线中载有1安培
的稳定电流,则在1秒内通过导线横截面的电量就被定义为1库仑(C)。也就
是说在此制下,我们有
\begin{equation}
    1C=1A\cdot s.
\end{equation}
必须指出,在MKSA制下,(7.1)式中各量的定义和单位已经给出,故$k$不 
能再任意指定而只能由计算求得,结果为$k=\widetilde{9}\times 10^9
(N\cdot m^2/C^2)$。为方便起见,在MKSA制下常将$k$写成
\begin{equation}
    k=\frac{1}{4\pi \varepsilon_0}
\end{equation}
的形式,相应的常量$\varepsilon_0$为 
\begin{equation}
    \varepsilon_0\approx 8.9\times 10^{-12}[C^2/(N\cdot m^2)].
\end{equation}
关于该常量的物理意义,后面第三章我们会提及。引入$\varepsilon_0$
后,式(7.1)就改写为
\begin{equation}
    F=\frac{1}{4\pi \varepsilon_0}\frac{q_1q_2}{r^2}.
\end{equation}

同一物理规律在不同单位制中可有不同的数学表达式。(1.5)和(1.9)式分别是
库仑定律在高斯制和MKSA制中的表达式。后者虽然比前者复杂,但由它推出的
许多关系式却比较简单。我们更倾向于采用MKSA制。
\subsection{库仑定律的矢量形式}
(7.1)式只反映了静电力的大小所服从的规律,并未涉及静电力的方向,
要反映方向就要把它改写为矢量形式。库仑定律的矢量形式可以表示为
\begin{equation}
    \boldsymbol{F}_{12}=\frac{q_1q_2}{4\pi \varepsilon_0r^2}
    \boldsymbol{e}_{r12},\quad\boldsymbol{F}_{21}
    =\frac{q_1q_2}{4\pi \varepsilon_0r^2}\boldsymbol{e}_{r21},
\end{equation}
其中,$\boldsymbol{F}_{12}$代表点电荷1对2的作用力,
$\boldsymbol{F}_{21}$代表点电荷2对1的作用力;
$\boldsymbol{e}_{r12}$代表由1指向2的单位矢量,
$\boldsymbol{e}_{r21}$代表由2指向1的单位矢量,
(显然$\boldsymbol{e}_{r12}=-\boldsymbol{e}_{r21}$)。
只要我们再规定正电荷的电荷量为正,负电荷的电荷量为负,
(1.10)式就可以很好地表示出静电力的方向性。

实验表明,在$10^{-15}$米至$10^{3}$米范围内库仑定律都成立。
这表明库仑力是长程力。
\subsection{叠加原理}
库仑定律讨论的是两个点电荷之间的静电力。当空间有两个以上点电荷
时,就必须补充另一实验事实——作用于每一电荷上的总静电力等于其他
点电荷单独存在时作用于该电荷的静电力的矢量和。这称为叠加原理,
它实际上意味着一个点电荷作用于令一点电荷的力总是符合库仑定律的,
不论周围是否存在其他电荷。库仑定律与叠加原理相配合,
原则上可以解决静电学的全部问题。
\section{静电场}
设空间存在静止的点电荷$Q$(如无特别声明,凡“静止”一律是相对于某个
事先选定的惯性系而言的),则人一点静止点电荷$q$必然受到来自$Q$的 
静电力,可见$Q$的存在使空间具有一种特殊的性质,我们说$Q$在周围
空间激发一个静电场。
\subsection{电场强度}
为了研究电场中各点的性质,可以用一个静止于该点的点电荷
(称为试探电荷)$q$做实验。试探电荷应该满足两个条件:

(i)其线度必须小到可被看作点电荷,以便确定场中每点的性质;

(ii)其电荷要足够小,使得它的置入不引起原有电荷的重新分布
(否则测出来的将是重新分布后的电荷激发的电场)。

先讨论静止点电荷$Q$激发的静电场。我们把在电场中所要研究的点
称为场点。在场点放置一个静止的试探电荷$q$。按照库仑定律,$q$
所受的电场力为
\begin{equation}
    \boldsymbol{F}=\frac{qQ}{4\pi \varepsilon_0 r^2}\boldsymbol{e}_r,
\end{equation}
其中$r$是场点与点电荷$Q$的距离,$\boldsymbol{e}_r$是从$Q$到$q$
的单位矢量。由(1.11)式我们可以看到,比值$\frac{\boldsymbol{F}}{q}$
是一个仅与场点有关的量,它可以表征场点的性质。这一结论还可以推广
至由任意静止电荷激发的电场,为此只需把激发电场的电荷分成许多
点电荷并利用叠加原理。我们把场中每点的$\frac{\boldsymbol{F}}{q}$
称为该点的电场强度(在近代文献中常又称为电场),以$\boldsymbol{E}$
代表,即 
\begin{equation}
    \boldsymbol{E}=\frac{\boldsymbol{F}}{q}.
\end{equation}
由这定义可知,电场强度是描写电场中某点性质的矢量,其大小等于
单位试探电荷在该点所受电场力的大小,其方向与正试探电荷在该点所
受电场力的方向相同。在场中任意指定一点,就有一个确定的电场强度
$\boldsymbol{E}$;对同一场中的不同点,$\boldsymbol{E}$一般可
以不同。各点的电场强度有相同大小和方向的电场称为均匀电场。

一般地说,若空间每点有一个标量$f$(例如地球周围每点有一个高度),
就说空间中存在一个标量场;若空间每点有一个矢量$\boldsymbol{a}$,
就说空间中存在一个矢量场。因为空间每点可用三个坐标$x,y,z$刻画,
所以标量场$f$和矢量场$\boldsymbol{a}$也可表示为坐标的函数
$f(x,y,z)$和$\boldsymbol{a}(x,y,z)$,于是标量场和矢量场又
称标量点函数和矢量点函数。电场强度是矢量场的一例,可以表示为
$\boldsymbol{E}(x,y,z)$。无论是标量场还是矢量场,都要特别注
意它作为坐标的函数的函数关系,“求某一带电体激发的电场”就是指
求出函数关系$\boldsymbol{E}(x,y,z)$。

电场强度的国际单位制单位由(7.12)定义,它没有专门名称,一
般记作$N/C$(或$V/m$,即伏/米,伏特的定义见后续内容)。
\subsection{电场线}
\subsubsection*{1、电场线的规定}
描写电场的最精确方法是写出电场强度$\boldsymbol{E}$的函数形式,
但这种描述不够直观。为了形象地描写电场,我们可以利用所谓电场线
的方法。首先,我们希望电场线能够表示电场的方向,故我们令电场线
满足条件:

(i)曲线上每点的切线方向与该点的电场强度方向相同;

另外,我们还希望电场线的疏密程度(即穿过某一与电场线垂直的单位
面元的电场线条数,下称其为电场线密度)能够反映电场强度的大小,
故我们再添加一规定,即

(ii)穿过场中任一面元的电场线条数正比于该面元的$\boldsymbol{E}$
通量\footnote{
    电场通量即电场与有向元面积的点乘,有向元面积的方向垂直于该面元。
},也就是说
\begin{equation}
    \text{通过任一}\Delta S\text{的电场线条数}=
    K\boldsymbol{E}\cdot \Delta \boldsymbol{S},
\end{equation}
其中$K$为比例常量$(K>0)$。

在此规定下,显然地有
\begin{equation}
    \text{电场线密度}=\frac{\Delta N}{\Delta S_\bot }
    =\frac{KE\Delta S_\bot}{\Delta S_\bot }=KE.
\end{equation}
其中$\Delta S_\bot$是指在元面积在电场方向的投影。
即电场线密度与电场强度的大小成正比,场线越密集,电场强度越大,
反之则电场强度越小。下面我们再介绍关于电场线的两个重要性质,
它们是由静电场的规律决定的。
\subsubsection*{2、电场线的性质}
\begin{theorem}
    性质1~~电场线发自正电荷(或无穷远),止于负电荷(或无穷远),在 
    无电荷处不中断。
    
    另外,电场线还具有另一条重要性质(这一性质涉及的电势概念将在后面提及):

    性质2~~电势沿电场线方向不断减小,因而电场线不构成闭合曲线。
\end{theorem}

这两条性质,在中学阶段我们均不予证明。
\subsection{电势能与电势}
在中学阶段,我们往往不会去真正地计算出电势能,常见的情况是给出电势能,
学生需根据电势能的含义去求解。考虑到这个原因,以及电势能的严格引入需要
一些微积分知识,我们就不再介绍电势能的计算,而是仅仅告知电势能和电势的含义。

实际上,我们在力学部分就已经接触过势能的概念,而电场中的所谓电势能,
与重力势能等没有任何分别,都是反映保守力做功的能力。在这里,我们也不再
论证电场力是保守力这一结论,但读者需时刻谨记,静电场力是保守力,静电场力 
做功与路径无关,仅与起始位置有关。

电势能的定义为:
\begin{definition}
    电荷在电场中某点的电势能在数值上等于把电荷从这点移到电势能为零处
    电场力所做的功。
\end{definition}

在理论研究中,通常取电荷$ q $在无限远处的电势能为零。电势能常常用$E_p$
或$V$或$U$表示。在SI制中,电势差的单位为$J/C$,这个单位有一个独特的名字,叫做 
“伏特”(V):
\begin{equation}
    1V=1J/C.
\end{equation}
在1V的电势差中移动1C的电荷量所需要做的功为1J。

如果经过计算,我们可以发现,电荷所在的电势能是一个仅与场点有关的
函数与其所带电荷量的成绩。为了表征电场本身的性质,我们除去电荷量$q$,定义 
\begin{equation}
    \phi=\frac{E_p}{q}
\end{equation}
为场点$P_0$处的电势,它是一个仅与场点有关的标量函数,因此是一个
标量场。电势叠加是标量叠加。

\section{静电场中的导体}
\subsection{导体和绝缘体}
整体而言,格雷和他的同伴们将物质分为了电学绝缘体和电学导体两部分。
它们之间的差别仍然是以自然性质的明显表现而划分的。一般的良导体
比如普通的金属与一般的绝缘体(比如玻璃和塑料)之间在电学传导性质
上相差$10^{20}$倍。

一个良导体和一个良好的绝缘体之间的电学差异就和液体和固体之间的力学
性质差异一样大。这完全不是偶然的。这两种性质都取决于原子微粒的流动
性:在电学方面,指的是电子质子等带电微粒的流动性,在力学性能
方面,指的是组成物质结构的原子或分子的流动性。可以把这个类比更加
深入一点,我们知道物质的物态可以随着温度的变化而变化,事实上,电学中
的电导率同样具有这样的特征,我们在良导体和良绝缘体中都发现了这样的例子,
并且某些物质可以根据其所在的条件,比如温度,在很大范围内改变自身的
电导率。其中很有趣也很有用的一类被称作半导体的材料就具有这样的性质。

基于某些原因,一种材料应该被看成是导体还是绝缘体应该和我们所关心的
那种现象的时间尺度有关。
\subsection{静电场中的导体}
在导体外施加一个恒定电场时,在电场作用下,导体内部的电荷重新分布,
导致总的电场发生变化,直到导体内静电力与非电力平衡,从而
形成一个稳定的静电状态,此时导体上不再存在电荷的流动。

在各向同性的均匀导体中,我们忽略非电力的影响,在导体中的电荷静
止时,显然电荷受到的静电力为零,故导体内部一定是没有电场的。
另外,由于导体表面的电荷也不流动,我们说导体表面的电场总是
垂直于此处面元的。显然,整个导体是等势的,因为电荷在导体上
无论怎么移动静电力都不会对它做功。

综合上面的论述,我们可以概括为:对于任何形状和排布方式的
导体系统,我们可以得出下面的结论:
\begin{theorem}
    (i)导体内部,$E=0$;\quad (ii)导体内部,$\rho=0$;
\end{theorem}

对于孤立导体,其电荷在表面突出尖锐处(即曲率大的地方)分布密集,在 
表面平坦处(曲率小的地方)分布较稀疏,在表面凹进去的地方(即曲率为负的
地方)分布最稀疏。其面电荷密度总体上与曲率正相关,但不是正比关系。

\subsection{唯一性定理与静电场中普遍问题的解决}
\subsubsection*{1、唯一性定理通俗描述}
\begin{theorem}
    所谓唯一性定理,通俗地讲,可以认为是这样一个定理:如果有一个电荷分布
    能够使得导体表面等势,那么这个电荷分布就是实际的电荷分布。
\end{theorem}

当然,上面这个叙述是非常不严格的,但是在中学阶段这个叙述完全够用。
注意,中学时候的静电屏蔽问题,需要依赖这一定理进行合理的说明。
往往高中的课程中对静电场中的导体含糊其辞,没有提供合理的推理依据,
会造成学生的极大疑惑,因此掌握利用唯一性定理解决导体问题的能力是
很重要的。

\subsubsection*{2、导体内部有空腔,但空腔内无电荷}
现在我们有一个内部有空腔的导体,在外界有一恒定的外加电场。
为了研究有空腔的导体,我们先从与该导体材质、外部形状大小
均相同的无空腔的导体开始研究。对于无空腔的导体,将它置于
相同的外加电场之下,显然稳定后外表面的电荷分布会使得导体
内部的电场为零。

现在我们考虑有空腔的导体。现在我们只需找出一种能够使其稳定
的电荷分布,根据唯一性定理,这种电荷分布就是实际的电荷分布。
由上面的无空腔导体的启发,我们使有空腔导体外表面分布有相同的电荷,
并且内表面不带电(即总电量为零且电荷均匀分布),显然,这种电荷分布
是稳定的,这就是实际的电荷分布。

我们可以将上述结论总结为:

(i)导体内表面不带电;

(ii)导体内部空腔内的电场为零。
\subsubsection*{3、导体空腔内有电荷且外表面接地}
现在我们的导体内部有一电荷$q$且外表面接地。外部无外加电场。
按照相同的思路,我们现在来探索其电荷分布情况。

如果空腔外面的导体无穷大(即除空腔外所有位置都铺满了整体不带电的导体),
那么此时空腔内表面会产生一种特定的电荷分布,它会使得外面的导体内没有电场。
并且我们围绕空腔取一个高斯面,显然通量为零,因此内电荷总量为零,即内表面
总带电量为$-q$。

现在空腔外面的导体是有限的,我们由无限的启发,只要将相同的电荷分布
复制到此时的空腔内表面上,外表面不加电荷,就可以使得外界电场为零,从而 
外表面与大地以及无穷远处等势,这符合了我们的实际要求。因此根据静电场
的唯一性定理,这就是实际的电场分布。

我们可以将上述内容概括为:

(i)有空腔的导体内部有电荷$q$且外表面接地时,导体仅内表面有电荷,且 
电荷量为$-q$。

(ii)导体外表面不带电,且导体外空间内电场为零。
\subsubsection*{4、导体空腔内有电荷且外表面不接地}
现在我们的导体内部有一电荷$q$且外表面不接地。外部无外加电场。
按照相同的思路,我们现在来探索其电荷分布情况。

此时我们仍然受到上面的启发,只要使得空腔的表面带与外部导体铺满空间时
相同的电荷$-q$,并且外表面电荷量为$q$且均匀分布(这使得外表面电荷不影
响内部电场),就可以达成稳定,再根据静电场的唯一性定理,这就是唯一的
电荷分布。

如果外表面是一个球面的话,这个导体在外界激发的电场与位于球心处带电量
为$q$的导体激发的电场相同。

以上内容可以概括为:
空腔内的电荷不会对外表面的电荷分布产生影响,或者空腔内的电场并未
“泄露”出空腔。即使有多个空腔,结论也是一样的,论证过程完全相同。
若是有外加电场,只需要将其与第一种情况组合一下就得到正确的
电荷分布,此处从略。
\section{电容器~电容}
\subsection{电容器}
\begin{definition}
    任何两个彼此绝缘而又互相靠近的导体,都可以看成是一个电容器。
\end{definition}

\begin{definition}
    使电容器带电叫做充电。充电时总是使电容器的一个导体带正电,另一个
    导体带等量的负电,每个导体所带电量的绝对值,叫做电容器所带的电量。
    
    使充电后的电容器失去电荷叫做放电,用一根导线把电容器的两极接通,
    两极上的电荷互相中和,电容器就不带电了。
\end{definition}

\subsection{电容}
\begin{definition}
    电容器所带的电量跟它的两极间的电势差的比值,叫做电容器的电容。
    如果用$Q$表示电容器所带的电量,用$ U $表示它的两极间的电势差,
    用$ C $表示它的电容,那么
    \begin{equation}
        C=\frac{Q}{U}
    \end{equation}
\end{definition}

在国际单位制里,电容的单位是法拉,简称法,国际符号
是F。一个电容器,如果带1库的电量时两极间的电势差是
1伏,这个电容器的电容就是1法。

\begin{equation}
    1F=1C/V
\end{equation}

法拉这个单位太大,实际上常用较小的单位:微法($\mu$F)
和皮法(pF)。它们间的换算关系是:
\begin{equation}
    1F=10^6\mu F=10^{12}pF
\end{equation}

\subsection{平行板电容器的电容}
\begin{theorem}
    对于一个平行板电容器,如果两板的正对面积为$S$,两板的距离为$d$,两板间充满
    介电常数为$\varepsilon$的电介质,那么,它的电容可以用下式来表示
    \begin{equation}
        C=\frac{\varepsilon\varepsilon_0S}{d}=\frac{\varepsilon S}{4\pi kd}
    \end{equation}
\end{theorem}

式中$S$用$m^2$作单位,$d$用$m$作单位,静电力恒量
$k=9\times 10^9N\cdot m^2/C^2$,算出的$C$以法为单位。
可以看出,平行板电容器的电容,跟介电常数成正比,跟正对面积
成正比,跟极板的距离成反比。

一般说来,电容器的电容是由两个导体的大小和形状、两
个导体的相对位置以及它们间的电介质定的。


