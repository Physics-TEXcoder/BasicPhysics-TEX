\section{磁场与电场是统一的}
历史上,电学和磁学最初完全是独立发展的,并且磁学的进展相当缓慢,
从这方面来看,高中教科书甚至很多普通物理教科书上将电学和磁学
分开来讲似乎是很合理的。传统的讲法,想必在任何一本教科书上都能够
找到,因此我们采取另外一种讲法。这种讲法并不是遵循历史发展规律的,
但是揭示了电学和磁学的联系,我们可以认为这种讲法是对传统讲法的一个
严格化,尽管在中学范围内我们并不能真正严格。下面的内容仅仅是一个
文字讲述,读者大概理解其意思即可,具体的数学计算不必考虑。

依据我们已有的电学理论,以及狭义相对论在尺度缩短方面的知识,我们有了计算运动
电荷对于静止电荷的作用力的方法。因此,我们在相对电荷静止的参考系内
计算出其他运动电荷对于该静止电荷的力,接着,我们利用相对论的洛伦兹变换,将该力
转换到相对该电荷运动的参考系内,从而会得到在该电荷运动的参考系内,该电荷
受到的力。

经过计算我们可以发现,在该电荷运动的参考系内,其受力比较复杂。
为了更简洁地描述运动电荷对于运动电荷的作用力,我们引入一个新的场
,这个场就叫做磁场。总的来说,磁场可以对运动的电荷产生力的作用。
运动的电荷可以产生磁场。从上面的叙述我们也可以知道,所谓磁场力,
不过是运动电荷对于运动电荷的电场力,电磁本质上是一致的。

但还有需要注意的一点,我们说电磁虽然本质上是统一的,但其统一性仅仅
在利用矢量分析的知识根据电来求解磁的方面得到了应用(实际历史上的求解磁场所
应用的仅仅是实验规律),而在后续的研究中,我们会将磁场作为一个比较独立的
场来研究,这也是没有问题的。

\section{磁场的作用力}
现在我们结束思想上的引领,回到具体的内容上来。尽管我们没有详细地说明,
但笔者可以告诉你,我们引入的磁场$\boldsymbol{B}$具有这样的特点,磁场
对于运动电荷产生的作用力为 
\begin{equation}
    \boldsymbol{F}=q\boldsymbol{v}\times \boldsymbol{B}
\end{equation}
关于向量积的内容可以查看第一章的向量部分。其中$\boldsymbol{B}$
被称为磁场的磁感应强度,其单位为特斯拉$(T)$。这种力也被称为洛伦兹力。

依据上面的公式,以及$\boldsymbol{I}=nqs\boldsymbol{v}$,我们可以得到磁场
对于电流元的作用力可以表示为
\begin{equation}
    \boldsymbol{F}=nq\Delta l\cdot s\cdot \frac{\boldsymbol{I}}{nqs}\times \boldsymbol{B}
    =\Delta l\boldsymbol{I}\times \boldsymbol{B}
\end{equation}
这种力也被称为安培力。实际上,电流这个单位的就是根据安培力量化的,
感兴趣可以去读相关资料,此处不再详述。
\section{法拉第电磁感应定律}
法拉第电磁感应定律在历史上是作为一个“定律”,也就是直接的实验规律被发现的,
但实际上,我们可以利用电学、相对论以及矢量分析的工具直接导出这一结论。
当然,我们不会在这里耗费巨大的篇幅去推导这些内容,一是考虑到读者很可能看不懂,
二是写起来比较麻烦。

\begin{theorem}
    法拉第电磁感应定律:电路中感生电动势的大小,跟穿过这一电路的磁通量的
    变化率成正比,即
    \begin{equation}
        \varepsilon=-\frac{\Delta \phi}{\Delta t}
    \end{equation}
\end{theorem}

\section{应用}
\subsection{直流电动机的反电动势}
电动机的线圈在磁场里转动时,线圈导线切割磁力线,所以在线圈中会产生
感生电动势。这个感生电动势的方向与外加电压的方向相反,通常把这个
电动势叫做反电动势。

正是感生电动势使得电动机与纯电阻电路不同。对于电动机来说,如果用$\varepsilon$
代表反电动势,则电流为 
\begin{equation}
    I=\frac{U-\varepsilon}{R}
\end{equation}
或者 
\begin{equation}
    UI=\varepsilon I+I^2R
\end{equation}
上式中的$ U I $是电路供给电动机的功率(输入功率),$\varepsilon I$是转化为机
械能的功率(输出功率),$I^2R$是在电动机线圈上损失的热功率。上式表示,电路供给电
动机的功率等于转化为机械能的功率与线圈上损失的热功率之和。从这里我们也看到
电功和电热的区别,在直流电动机中,由于存在着反电动势,有一部分电能转化为
机械能,电功并不等于电热,而是大于电热。
\subsection{自感}
\begin{definition}
    由于导体本身的电流发生变化而产生的电磁感应现象,叫做自感现象。
    在自感现象中产生的感生电动势,叫做自感电动势。
\end{definition}

自感电动势跟其他感生电动势一样,是跟穿过线圈的磁通量的变化率
$\Delta\phi/\Delta t$ 成正比的,我们知道,磁通量$\phi$跟磁感应强度
$B$成正比,$B$又跟产生这个磁场的电流$I$成正比. 所以$\phi$跟$I$成
正比,$\Delta\phi$跟$\Delta I$也成正比,由此可知自感电动势
$\mathcal{E}=\Delta\phi/\Delta t$跟$\Delta I/\Delta t$成正比,即
\begin{equation}
    \mathcal{E}=L\frac{\Delta I}{\Delta t}
\end{equation}
式中的比例恒量$L$叫做线圈的自感系数,简称自感或电感,它是由线圈本身的
特性决定的。线圈越长,单位长度上的匝数越多,截面积越大,它的自感系数就越大,
另外,有铁芯的线图的自感系数,比没有铁芯时要大得多。对于一个现成的线圈来说,
自感系数是一定的。

自感系数的单位是亨利,简称亨,国际符号是 H。如果通过线圈的电流强度在 1 秒
钟以改变 1 安时产生的自感由动势是 1 伏,这个线圈的自感系数是 1亨,所以
\begin{equation}
    1\mathrm H=1\mathrm V\cdot\mathrm s/\mathrm A
\end{equation}