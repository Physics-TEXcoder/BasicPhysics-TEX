\section{机械振动}
\subsection{机械振动}
\begin{definition}
    物体(或者物体的一部分)在平衡位置附近来回做往复运动,叫做机械振
    动,常常简称为振动。
\end{definition}

物体之所以能够形成机械振动,是因为受到指向平衡位置的力的作用,使
振动物体回到平衡位置的力叫做回复力。
\subsection{表征振动的物理量}
\begin{definition}
    振动物体离开平衡位置的最大距离叫做振幅;\\
    完成一次全振动经过的时间叫做周期;\\
    在 1 秒内完成全振动的次数叫做频率;频率的单位是赫兹,简称赫,国际符号是 Hz。
\end{definition}

如果用$T$(秒)表示周期,用$f$(赫)表示频率,那么
\begin{equation}
    f=\frac{1}{T},~~T=\frac{1}{f}
\end{equation}
振幅、周期或频率是表征整个振动的物理量。一个振动,如果知道了它的
振幅、周期或频率,我们就从整体上把握了振动的情况。
\subsection{简谐振动}
让我们考虑一个与弹簧连在一起的质点在光滑桌面上的运动。假设 
弹簧的另外一端固定在墙壁上,而其质量可以忽略不计。若我们将 
弹簧未变形时质点的坐标取作$x=0$,根据Hooke定律,当质点运动到
$x$处时质点所受到的力为
\begin{equation}
    \boldsymbol{F}=-kx\boldsymbol{i}.
\end{equation}
其运动方程为
\begin{equation}
    m\frac{d^2x}{dt^2}=-kx,
\end{equation}
或是
\begin{equation}
    \frac{d^2x}{dt^2}+\frac{k}{m}x=
    \frac{d^2x}{dt^2}+\omega_0^2=0.
\end{equation}
这里,$\omega_0=\sqrt{\frac{k}{m}}$称为弹簧振子的本征频率。

下面,我们再来考虑一个单摆的运动方程。我们看到,拴在长度为$l$的 
摆线下端的小物体所受到的外力有二:一是重力$-mg\boldsymbol{j}$,二是
摆线对其的拉力$\boldsymbol{T}$。显然,小物体做的是一个以$O$点为圆心,
$l$为半径的圆周运动。我们选取$O$点为支点,根据转动定理\footnote{
    这个知识已经超出中学范围,作为了解即可。
},我们有 
\begin{equation}
    -mglsin\theta =ml^2\frac{d^2\theta }{dt^2 }.
\end{equation}
当小物体在其平衡点附近做微小振动时,我们近似地有$sin\theta\approx \theta $
。因此,上式可以近似地写作
\begin{equation}
    ml\frac{d^2\theta }{dt^2 }+mgsin\theta\approx
    ml\frac{d^2\theta }{dt^2 }+mg\theta=0,
\end{equation}
或是
\begin{equation}
    \frac{d^2\theta }{dt^2 }+\tilde{\omega}^2\theta =0.
\end{equation}
这里,$\tilde{\omega}=\sqrt{\frac{g}{l}}$称为单摆的振动频率。

结合上面两个例子,我们看到,有许多振动体系的运动可以由
下面的微分方程
\begin{equation}
    \frac{dy(t)}{dt^2}+\Omega^2y(t)=0
\end{equation}
来描述。这一方程称为简谐振动方程,容易验证,它的通解可以写作
\begin{equation}
    y(t)=Acos(\Omega t+\varphi_0).
\end{equation}
而相应的运动被称之为简谐振动。依照惯例,$\Omega$被称为 
该体系的固有振动频率。它代表振子的相位在单位时间内的改变量。
相应地,振子的振动周期定义作
\begin{equation}
    T=\frac{2\pi}{\Omega}.
\end{equation}
而$A$和$\varphi_0$则是两个需要通过振动的初始条件来确定的量,
被分别称为振子的振幅和初相位。$A$一般取一个正数,而$\varphi_0$
的取值范围则为$(0\leqslant \varphi_0\leqslant 2\pi)$。

根据式(4.10),我们可以得到弹簧振子和单摆的振动周期分别为
\begin{equation}
    T_{\text{弹簧}}=2\pi\sqrt{\frac{m}{k}},~~T_{\text{单摆}}=2\pi\sqrt{\frac{l}{g}}
\end{equation}
在中学阶段我们不要求上面这两个公式的推导,读者只需学会应用即可。
\subsection{相和相差}
\begin{definition}
    两个频率相同的简谐振动,如果它们的振动步调一致,我们就说它们的相
    是相同的,简称为同相;
    多数情况下,两个简谐振动的步调并不一致,即它们的相不同,或者说它
    们存在着相差;其中有一种特殊情形,即振动步调正好相反,这种情形叫做反相。
\end{definition}

\subsection{阻尼振动}
简谐振动不考虑阻力,振动系统的机械能是守恒的。简谐振动是一种理想化的振动,
一旦供给振动系统一定的能量来使它开始振动,由于机械能守恒,它就要以一定的
振幅永不停息地振动下去。可是实际上振动系统不可避免地要受到摩擦和其他阻力,
即受到阻尼的作用。
\begin{definition}
    振动系统克服阻尼作用做功,系统的机械能就要损耗,这样,机械能随着时间逐渐减少,
    振动的振幅也逐渐减小。待到机械能耗尽之时,振动就停下来了。这种振幅逐
    渐减小的振动叫做阻尼振动。
\end{definition}

振动系统受到的阻尼越大,振幅减小得越快,振动停下来也越快。阻尼过
大将不能产生振动。阻尼越小,振幅减小得越慢。在阻尼很小时,在一段不太
长的时间内看不出振幅有明显的减小,就可以把振动系统当作简谐振动来处理。

\subsection{受迫振动和共振}
\begin{definition}
    物体在周期性外力作用下的振动叫做受迫振动。
\end{definition}

\begin{theorem}
    物体做受迫振动的频率等于驱动力的频率,而跟物体的固有频率没有关系。
\end{theorem}

\begin{theorem}
    当驱动力的频率跟物体的固有频率相等的时候,受迫振动的振幅最大,这
    种现象叫做共振。
\end{theorem}

关于阻尼振动、受迫振动和共振相关的知识远远超出了中学物理的范畴,因此
仅仅提及,不作为重点。

\section{机械波}
\subsection{机械波及其分类}
\begin{definition}
    机械振动在媒质中的传播叫做机械波。
\end{definition}

\begin{definition}
    振动方向与波的传播方向垂直的波叫做横波;\\
    振动方向与波的传播方向在同一直线上的波叫做纵波。
\end{definition}

\subsection{机械波的基本属性}
\begin{definition}
    沿着波的传播方向,两个相邻的同相质点间的距离,叫做波长,波长通常
    用字母$\lambda$表示。
\end{definition}

在一个周期$T$内,振动传播的距离等于波长$\lambda$,那么振动传播的速率
——波速$v$可以根据下面的式子求出:
\begin{equation}
    v=\frac{\lambda}{T}
\end{equation}
由于$T=1/f$,因此
\begin{equation}
    v=\lambda f
\end{equation}
这个关系,虽然我们是从机械波得到的,但是它对于今后将要学习的电磁波、光波
也是适用的。机械波在媒质中传播的速率是由媒质本身的性质决定的,在不同媒质中传
播的速率并不相同。
\subsection{波的干涉}
\subsubsection*{1、波的叠加}
\begin{theorem}
    几列波相遇时能够保持各自的状态而不互相干扰,这是波的一个基本性质;
    两列波重叠区域里任何一点的总位移,都等于两列波分别引起的位移的向量和。
\end{theorem}

\subsubsection*{2、波的干涉}
\begin{definition}
    频率相同的两列波叠加,使某些区域的振动加强,集些区域的振动减弱,并
    且振动加强和振动减弱的区域互相间隔,这种现象叫波的干涉,形成的图样叫
    干涉图样。要得到稳定的干涉现象,观察到干涉
    图样,两个波源必须是频率相同、相差恒定,这样的波源叫相干波源。
\end{definition}

\subsection{波的衍射}
\begin{definition}
    波绕过障碍物的现象,叫做波的衍射;能够发生明显的衍射现象的条件是:障
    碍物或孔的尺寸跟波长相差不多。一切波都发生衍射,衍射也是波的特有现象。
\end{definition}