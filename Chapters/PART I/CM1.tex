\section{力}
\subsection{力的概念}
\begin{definition}
    我们将物体对物体的作用称作力。
\end{definition}

很显然地,我们认为力是有大小的,事实上,
在国际单位制下,力的单位是牛顿,简称牛,国际符号是N。并且,我们很容易
地知道,描述一个力是需要方向的,例如,你向不同的方向推车,即对车施加
作用,产生的效果是不相同的,如果没有方向的概念,我们将无法区分这两个
作用,或者说这两个力。

为了直观地说明力的作用,我们常常用一根带箭头的线段来表示力。
这个线段的长度代表力的大小,指向表示力的方向。
\subsection{重力}
\begin{definition}
    地球上一切物体都受到地球的吸引作用。这种由于地球的吸引而使得
    物体受到的力叫做重力。
\end{definition}

一个物体的各部分都要受到地球对它的作用力,我们可以认为重力的作用
集中于一点,这一点叫做物体的重心。或者说,我们可以把重力总的作用
等效于作用在中心上的等大的力,实际上,后面我们会知道,重心上的
等效力需要满足重力在大小和力矩上的等效,我们实际也是通过这种方式
确定重心的位置的。
\subsection{弹力}
\begin{definition}
    发生形变的物体,由于要恢复原状,对跟它接触的物体会产生力的作用,
    这种力叫做弹力。
\end{definition}

弹力的方向是与形变方向相反的。

\subsection{胡克定律}
弹力的大小根形变的大小有关,形变越大,弹力也越大。
\begin{theorem}
    弹簧弹力的大小$f$和弹簧伸长(或缩短)的长度$x$成正比,写成公式就是
    \begin{equation}
        f=kx
    \end{equation}
    其中$k$是比例系数,称为弹簧的劲度系数或者倔强系数。
    倔强系数是一个有单位的量。在国际单位制中,$f$ 的单位是牛,
    $x$ 的单位是米,$k$ 的单位是牛/米。
\end{theorem}
这个规律是由英国科学家胡克发现的,叫做胡克定律。

胡克定律有它的适用范围。物体的形变过大,超出一定限度,上述正比关
系将不再适用,这时即使撤去外力,物体也不能完全恢复原状。这个限度叫做
弹性限度。胡克定律在弹性限度内适用。弹性限度内的形变叫做弹性形变。本
书中提到的形变,除非特别指明,一般是指弹性形变。

\subsection{摩擦力}
\begin{definition}
    当一个物体在另一个物体表面上做相对滑动的时候,要受到另一个物
    体阻碍它运动的力,这种力叫做滑动摩擦力。滑动摩擦力的方向总跟
    接触面相切,并且跟物体的相对运动的方向相反。
\end{definition}

\begin{theorem}
    实验表明:滑动摩擦力跟压力成正比,或者说,跟一个物体对另一个物体
    表面的垂直作用力成正比。用$f$表示滑动摩擦力的大小,$N$表示压力的
    大小,那么 
    \begin{equation}
        f=\mu N
    \end{equation}
    其中$\mu$是比例常数,叫做滑动摩擦系数。这是一个比值,没有单位。
\end{theorem}

\subsection{牛顿第三定律}
\begin{theorem}
    两个物体之间的作用力和及作用力总是大小相等,方向相反,作用
    在一条直线上。这就是牛顿第三定律。
\end{theorem}

有人喜欢将其简称为“牛三”,我个人是不认同这种称呼的。一般地,我们 
将其简称为“第三定律”是比较合适的。
\subsection{力的合成与分解}
\begin{definition}
    一个力,如果它产生的效果跟几个力共同产生的效果相同,这个力就
    叫做那几个力的合力,求几个力的合力叫做力的合成。
\end{definition}

\begin{theorem}
    实验表明,对于两个互成角度的共点力,可以用表示这两个力的线段为邻
    边作平行四边形,这两个邻边之间的对角线就表示合力的大小和方向,
    这叫做力的平行四边形法则。
\end{theorem}

\begin{definition}
    如果几个力产生的效果跟原来一个力产生的效果相同,这几个力就叫做原
    来那个力的分力。求一个已知力的分力叫做力的分解。
\end{definition}

\section{向量}
力不但有大小,而且有方向。相同大小的力,方向不同,它们的作用效果并
不相同。要把一个力完全表达出来,除了说明它的大小,还要指明它的方向才
行。

这种既有大小又有方向的物理量,除了力而外,在物理学中还有很多。我
们在初中学过的速度也是这类物理量。

这样,我们就接触到一类物理量。它们的共同特点是:既有大小,又有方
向。\begin{definition}
    这种既有大小又有方向的物理量,叫做向量。
\end{definition}

力是矢量,速度也是向量。那
些只有大小没有方向的物理量,叫做标量。长度、质量、时间是标量,初中学
过的功、温度等也是标量。
\subsection{向量的加减规则}
我们知道,力的描述需要大小和方向两个量。为了描述力,
我们用有方向的线段代表力,该线段的长度在数值上等于
力的大小,箭头指向即为力的方向。

我们可以把这种有方向的线段叫做向量,把线段的长度叫做向量的模,
向量$\boldsymbol{a}$的模被记为$|\boldsymbol{a} | $或$a$,
并把模等于0的向量称为零向量,零向量仅仅是为向量体系的完整性而
提出的概念,它的的方向是任意的。

在这种描述方法下,当物体受到一个力,又受到第二个力,
那么在几何上,显然地,质点的总受力可以由代表分力的向量
首尾相接后从末端指向前端的第三个向量来表示。

\begin{definition}
    简洁起见,我们可以定义
    \begin{equation}
        \boldsymbol{a}+\boldsymbol{b}
    \end{equation}
    为两个向量按次序首尾相接后的从末端指向前端的向量,其中
    $\boldsymbol{a}$代表第一个向量,$\boldsymbol{b}$代表第二个向量,
    这样,$\boldsymbol{a}+\boldsymbol{b}$所表示的向量就可以代表
    由向量$\boldsymbol{a}$和$\boldsymbol{b}$代表的力的合力。
\end{definition}

有了“$+$”这种记号,或者说运算方式,我们可以将一些显而易见的
规律表示为
\begin{equation}
    \boldsymbol{a}+\boldsymbol{b}=
    \boldsymbol{b}+\boldsymbol{a}.
\end{equation}
以及
\begin{equation}
    (\boldsymbol{a}+\boldsymbol{b})+\boldsymbol{c}=
    \boldsymbol{a}+(\boldsymbol{b}+\boldsymbol{c}).
\end{equation}
这两条规律分别被称为向量相加的交换律和结合律。

另外,有时我们还会涉及从力中消去一段力,因此我们用 
\begin{equation}
    \boldsymbol{a}-\boldsymbol{b}
\end{equation}
表示能够代表原始力消去一段力后的剩余力的向量,其中
向量$\boldsymbol{a}$代表原始力,向量$\boldsymbol{b}$
代表消除的力。

我们很容易知道
\footnote{如果你觉得不容易知道的话可以亲自来问作者,我懒得在这里
详细叙述了。邮箱在封面上已经给出,可通过邮件联系。}
,$\boldsymbol{a}-\boldsymbol{b}$代表的向量实际上
等同于向量$\boldsymbol{a}$和与$\boldsymbol{b}$向量方向相反
但长度相等的向量相加后得到的向量。
因此若我们用$-\boldsymbol{b}$代表与$\boldsymbol{b}$向量方向相反
但长度相等的向量,并且使用记号(1.3),我们可以将这一规律表示为 
\begin{equation}
    \boldsymbol{a}-\boldsymbol{b}
    =\boldsymbol{a}+(-\boldsymbol{b}).
\end{equation}
以上就是向量体系的加减运算规则。

值得注意的是,尽管我们是通过力引进了这一基本规定,
但它对于像速度、加速度等物理量都是适用的,我们也可以
利用向量来描述这些物理量。

\subsection{向量与数的乘法}\footnote{
    目前我们还没有看到从这一部分之后的内容的实际意义,但为了完备性,
    我们提前给出这些定义。读者若无法理解其中的情境,可在学习后续课程后
    再来理解。
}
显然地,仅仅只有上面的所谓向量加减运算还不足以使得向量
成为描述带方向物理量的实用体系。在考虑力的方向的前提下,
牛顿第二定律告诉我们,一质点所受力的向量等同于长度为质点
质量倍,方向与质点的加速度同向的向量,但是再未补充新的
定义之前,我们无法用向量的表达式表现这一规律。

\begin{definition}
    我们定义$\lambda \boldsymbol{a}$表示一个向量,
    其中$\boldsymbol{a}$是一个向量,$\lambda$是一个实数。
    并且它的模满足
    \begin{equation}
        | \lambda \boldsymbol{a}|= |\lambda  | |\boldsymbol{a}|,
    \end{equation}
    它的方向当$\lambda >0$时与$\boldsymbol{a}$同向,反之则反向;
    而当$\lambda =0$时,该向量为所谓零向量。
\end{definition}

那么在这一定义下,牛顿第二定律的矢量形式可以用向量简洁地描述为 
\begin{equation}
    \boldsymbol{F}=m\boldsymbol{a}.
\end{equation}
在之前的定义下,我们可以很容易地知道
\begin{equation}
    \lambda(\mu\boldsymbol{a})=\mu(\lambda\boldsymbol{a})
    =(\lambda \mu)\boldsymbol{a}.
\end{equation}
因为这三个式子所代表的向量的大小和方向都是相同的,这是显而
易见的。这条规则被称为向量数乘的结合律。

我们还很容易地知道
\begin{equation}
    (\lambda+\mu)\boldsymbol{a}=\lambda\boldsymbol{a}+
    \mu\boldsymbol{a}.
\end{equation}
这条规则被称为向量数乘的分配律。

向量相加及数乘的运算统称为向量的线性运算。
\subsection{使用空间直角坐标系简化向量的线性运算}
我们在三维空间中建立直角坐标系,并引入三个长度为1的
分别与$x,y,z$轴平行的向量$\boldsymbol{i},\boldsymbol{j},
\boldsymbol{k}$。那么,依据向量相加的交换律与结合律和向量
数乘的结合律与分配律,我们很容易地知道,以空间中$A(x_1,y_1,z_1)$和
$B(x_2,y_2,z_2)$为起始点的向量$\boldsymbol{AB}$满足
\begin{equation}
    \boldsymbol{AB}=(x_2-x_1)\boldsymbol{i}+
    (y_2-y_1)\boldsymbol{j}+(z_2-z_1)\boldsymbol{k}
    =\Delta x\boldsymbol{i}+\Delta y\boldsymbol{j}
    +\Delta z\boldsymbol{k}.
\end{equation}
并且,若我们再任意引入一个向量
\begin{equation}
    \boldsymbol{CD}=\Delta x'\boldsymbol{i}
    +\Delta y'\boldsymbol{j}+\Delta z'\boldsymbol{k}.
\end{equation}
再次依据之前的定义和规律,两向量的和可以被写为
\begin{equation}
    \boldsymbol{AB}+\boldsymbol{CD}=(\Delta x'+\Delta x)\boldsymbol{i}
    +(\Delta y'+\Delta y)\boldsymbol{j}
    +(\Delta z'+\Delta z)\boldsymbol{k}.
\end{equation}
并且向量的数乘可以被写作
\begin{equation}
    \lambda \boldsymbol{AB}=(\lambda \Delta x)\boldsymbol{i}
    +(\lambda \Delta y)\boldsymbol{j}
    +(\lambda \Delta z)\boldsymbol{k}.
\end{equation}
也就是说,对向量进行加减以及与数相乘,只需对向量的各个坐标分别进行
相应的数量运算就可以了。

另外,在物理里面,为了描述空间中点的位置,我们常常使用所谓向径或
矢径。向径即为起始点为原点,终点为点所在位置坐标的向量。例如,
当点位于$E(x,y,z)$处时,其向径为
\begin{equation}
    \boldsymbol{r}=x\boldsymbol{i}+y\boldsymbol{j}
    +z\boldsymbol{k}.
\end{equation}
向径的定义在物理中也有着重要的应用,但没有新的根本性的内容,
我们不再赘述。

\subsection{向量的投影}
\begin{definition}
    我们用
    $\langle \boldsymbol{a},\boldsymbol{c}\rangle$
    表示向量$\boldsymbol{a},\boldsymbol{c}$之间的夹角,其取值
    小于$\pi$。并且定义
    \begin{equation}
        Prj_c\boldsymbol{a}=|\boldsymbol{a} |
        \cos\langle \boldsymbol{a},\boldsymbol{c}\rangle  
    \end{equation}
    为向量$\boldsymbol{a}$在向量$\boldsymbol{c}$上的投影。
\end{definition}
那么,显然投影存在性质
\begin{equation}
    Prj_c(\boldsymbol{a}+\boldsymbol{b})=
    Prj_c\boldsymbol{a}+Prj_c\boldsymbol{b}
\end{equation}
以及 
\begin{equation}
    Prj_c(\lambda\boldsymbol{a})=\lambda
    Prj_c\boldsymbol{a}.
\end{equation}
\subsection{新定义-点乘}
我们知道,在力学中有着所谓做功的概念,即力$\boldsymbol{F}$
作用于质点上,并且质点移动了$d\boldsymbol{r}$的过程中,该 
力对质点所做的功可以用向量表示为
\begin{equation}
    dW=| \boldsymbol{F}| |d\boldsymbol{r} | \cos\theta .
\end{equation}
其中$\theta $为向量$\boldsymbol{F}$和向量$d\boldsymbol{r}$
之间的夹角。

\begin{definition}
    现在我们定义
    \begin{equation}
        \boldsymbol{a}\cdot \boldsymbol{b}=
        | \boldsymbol{a}| |\boldsymbol{b} | \cos\theta.
    \end{equation}
    其中$\theta $是向量$\boldsymbol{a},\boldsymbol{b}$之间的夹角。
    这种新定义出来的运算方式被称为向量的点乘,运算的结果被称为两个向量
    的数量积。
\end{definition}

定义了这种新运算后,显然地,我们可以用这种新运算来表示
做功等物理量,即做功可以被表示为
\begin{equation}
    dW=\boldsymbol{F}\cdot d\boldsymbol{r}
    | \boldsymbol{F}| |d\boldsymbol{r} | \cos\theta .
\end{equation}
实际上,定义这种新运算不过是相当于给原来的式子换了一种更简洁
更利于研究的记号而已。容易知道,数量积满足
\begin{equation}
    \boldsymbol{a}\cdot \boldsymbol{b}=\boldsymbol{b}
    \cdot \boldsymbol{a}.
\end{equation}
这一规则常被称为交换律,易证不再赘述。另外,数量积还满足
\begin{equation}
    (\boldsymbol{a}+\boldsymbol{b})\cdot \boldsymbol{c}
    =\boldsymbol{a}\cdot \boldsymbol{c}
    +\boldsymbol{b}\cdot \boldsymbol{c}.
\end{equation}
这一规则常被称为分配律,可以作如下证明:

\begin{equation}
    (\boldsymbol{a}+\boldsymbol{b})\cdot \boldsymbol{c}=
    | \boldsymbol{c}|Prj_c(\boldsymbol{a}+\boldsymbol{b}) .
\end{equation}
而投影有性质
\begin{equation}
    Prj_c(\boldsymbol{a}+\boldsymbol{b})=
    Prj_c\boldsymbol{a}+Prj_c\boldsymbol{b}.
\end{equation}
那么 
\begin{equation}
    (\boldsymbol{a}+\boldsymbol{b})\cdot \boldsymbol{c}=
    | \boldsymbol{c}|(Prj_c\boldsymbol{a}+Prj_c\boldsymbol{b})
    =\boldsymbol{a}\cdot \boldsymbol{c}
    +\boldsymbol{b}\cdot \boldsymbol{c}.
\end{equation}

数量积的第三个规律被称为结合律,即 
\begin{equation}
    (\lambda\boldsymbol{a})\cdot \boldsymbol{b}
    =\lambda (\boldsymbol{a}\cdot \boldsymbol{b}).
\end{equation}
其中$\lambda$为实数。当$\boldsymbol{b}=\boldsymbol{0}$时,
上式显然成立;当$\boldsymbol{b}\neq \boldsymbol{0}$时,按
投影的性质,我们可以知道 
\begin{equation}
    (\lambda\boldsymbol{a})\cdot \boldsymbol{b}
    =| \boldsymbol{b}|Prj_b(\lambda \boldsymbol{a})
    =| \boldsymbol{b}|\lambda Prj_b\boldsymbol{a}
    =\lambda | \boldsymbol{b}|Prj_b\boldsymbol{a}
    =\lambda (\boldsymbol{a}\cdot \boldsymbol{b}).
\end{equation}
由上述结合律,利用交换律,还容易推得
\begin{equation}
    \boldsymbol{a}\cdot (\lambda \boldsymbol{b})=
    \lambda (\boldsymbol{a}\cdot \boldsymbol{b})
\end{equation}
以及 
\begin{equation}
    (\lambda \boldsymbol{a})\cdot (\mu \boldsymbol{b})=
    \lambda \mu(\boldsymbol{a}\cdot \boldsymbol{b}).
\end{equation}
这是因为
\begin{equation}
    \boldsymbol{a}\cdot (\lambda\boldsymbol{b})=
    (\lambda\boldsymbol{b})\cdot \boldsymbol{a}=
    \lambda(\boldsymbol{b}\cdot \boldsymbol{a})=
    \lambda (\boldsymbol{a}\cdot \boldsymbol{b}),
\end{equation}
\begin{equation}
    (\lambda \boldsymbol{a})\cdot (\mu \boldsymbol{b})=
    \lambda [\boldsymbol{a}\cdot (\mu\boldsymbol{b})]=
    \lambda [\mu(\boldsymbol{a}\cdot \boldsymbol{b})]=
    \lambda \mu(\boldsymbol{a}\cdot \boldsymbol{b}).
\end{equation}
\subsection{数量积的坐标表示}
在空间直角坐标系下,有向量 
\begin{equation}
    \boldsymbol{a}=a_x\boldsymbol{i}+a_y\boldsymbol{j}
    +a_z\boldsymbol{k},~~~\boldsymbol{b}=b_x\boldsymbol{i}
    +b_y\boldsymbol{j}+b_z\boldsymbol{k}.
\end{equation}
容易知道,其数量积为
\begin{equation}
    \boldsymbol{a}\cdot \boldsymbol{b}=a_xb_x+a_yb_y+a_zb_z.
\end{equation}
运用之前的规律很容易就能推出,因而我们不再赘述推导过程。

另外,由于
\begin{equation}
    \boldsymbol{a}\cdot \boldsymbol{b}
    =| \boldsymbol{a}| |\boldsymbol{b} | \cos\theta ,
\end{equation}
所以当$\boldsymbol{a},\boldsymbol{b}$都不是零向量时,有 
\begin{equation}
    \cos\theta=\frac{\boldsymbol{a}\cdot \boldsymbol{b}}
    {| \boldsymbol{a}| |\boldsymbol{b} | }.
\end{equation}

将数量积的坐标表示式及向量的模的坐标代入上式,就得到
\begin{equation}
    \cos\theta =\frac{a_xb_x+a_yb_y+a_zb_z}
    {\sqrt{a_x^2+a_y^2+a_z^2}\sqrt{b_x^2+b_y^2+b_z^2}},
\end{equation}
这就是两向量夹角余弦的坐标形式。
\subsection{新定义-叉乘}
当研究物体转动时,我们有所谓力矩
\begin{equation}
    | \boldsymbol{M}| =| \boldsymbol{r}| |\boldsymbol{F} |\sin\theta .
\end{equation}
其中$\theta $为向量$\boldsymbol{r}$与$\boldsymbol{F}$的夹角。
我们知道,力矩显然是具有方向性的,即使力和向径的大小、夹角都相同,
作用效果也常常不同。在物理中,我们规定力矩的方向由右手螺旋的规律决定。

\begin{definition}
    在物理的基础上,我们规定
    \begin{equation}
        \boldsymbol{a}\times \boldsymbol{b}
    \end{equation}
    代表大小为$| \boldsymbol{a}| |\boldsymbol{b} |\sin\theta $,
    方向遵从右手螺旋规律的向量。
\end{definition}

有了这种新的写法,我们可以将力矩写为
\begin{equation}
    \boldsymbol{M}=\boldsymbol{r}\times\boldsymbol{F}.
\end{equation}

并且,我们有以下规律:
\begin{equation}
    \boldsymbol{b}\times\boldsymbol{a}=-\boldsymbol{a}\times
    \boldsymbol{b},
\end{equation}
\begin{equation}
    (\boldsymbol{a}+\boldsymbol{b})\times \boldsymbol{c}
    =\boldsymbol{a}\times \boldsymbol{c}+\boldsymbol{b}\times
    \boldsymbol{c},
\end{equation}
以及 
\begin{equation}
    (\lambda\boldsymbol{a})\times\boldsymbol{b}=
    \boldsymbol{a}\times (\lambda \boldsymbol{b})
    =\lambda(\boldsymbol{a}\times \boldsymbol{b}).
\end{equation}
后两个规律分别被称为分配律和结合律。
\subsection{向量积的坐标表示}
在空间直角坐标系下,有向量
\begin{equation}
    \boldsymbol{a}=a_x\boldsymbol{i}+a_y\boldsymbol{j}
    +a_z\boldsymbol{k},~~~\boldsymbol{b}=b_x\boldsymbol{i}
    +b_y\boldsymbol{j}+b_z\boldsymbol{k}.
\end{equation}
那么,按照上述运算规律,其向量积为
\begin{equation}
    \begin{aligned}
        \boldsymbol{a}\times \boldsymbol{b} 
        =&(a_x\boldsymbol{i}+a_y\boldsymbol{j}+a_z\boldsymbol{k})\times(b_x\boldsymbol{i}+b_y\boldsymbol{j}+b_z\boldsymbol{k})  \\
        =&a_x\boldsymbol{i}\times(b_x\boldsymbol{i}+b_y\boldsymbol{j}+b_z\boldsymbol{k})+a_y\boldsymbol{j}\times(b_x\boldsymbol{i}+b_y\boldsymbol{j}+b_z\boldsymbol{k})+a_z\boldsymbol{k}\times(b_x\boldsymbol{i}+b_y\boldsymbol{j}+b_z\boldsymbol{k}) \\
        =&a_xb_x(\boldsymbol{i}\times\boldsymbol{i})+a_xb_y(\boldsymbol{i}\times\boldsymbol{j})+a_xb_z(\boldsymbol{i}\times\boldsymbol{k})+a_yb_x(\boldsymbol{j}\times\boldsymbol{i})+a_yb_y(\boldsymbol{j}\times\boldsymbol{j})+a_yb_z(\boldsymbol{j}\times\boldsymbol{k})+ \\
        &a_zb_x(\boldsymbol{k}\times\boldsymbol{i})+a_zb_y(\boldsymbol{k}\times\boldsymbol{j})+a_zb_z(\boldsymbol{k}\times\boldsymbol{k}).
    \end{aligned}
\end{equation}
因为 
\begin{equation}
    \begin{aligned}
        &\boldsymbol{i}\times \boldsymbol{i}=\boldsymbol{j}\times \boldsymbol{j}
        =\boldsymbol{k}\times \boldsymbol{k}=0,
        ~~\boldsymbol{i}\times \boldsymbol{j}=\boldsymbol{k},~~
        \boldsymbol{j}\times \boldsymbol{k}=\boldsymbol{i},~~\\
        &\boldsymbol{k}\times \boldsymbol{i}=\boldsymbol{j},~~
        \boldsymbol{j}\times \boldsymbol{i}=-\boldsymbol{k},~~
        \boldsymbol{k}\times \boldsymbol{j}=-\boldsymbol{i},~~
        \boldsymbol{i}\times \boldsymbol{k}=-\boldsymbol{j},
    \end{aligned}
\end{equation}
所以 
\begin{equation}
    \boldsymbol{a}\times\boldsymbol{b}=
    (a_yb_z-a_zb_y)\boldsymbol{i}+
    (a_zb_x-a_xb_z)\boldsymbol{j}+
    (a_xb_y-a_yb_x)\boldsymbol{k}.
\end{equation}
这就是向量积的坐标表示。
\section*{结语}
理解向量的关键就在于,将向量的体系看作是描述物理中矢量的
工具,而不是直接地将向量看作具有大小和方向的物理量本身。

首先,我们用有方向的线段(即向量)描述需要彰显方向的物理量(矢量),
然后我们根据物理的需求,给向量带来了四种基本的运算方式规定,
即向量的相加规定、向量与数相乘的规定、向量的点乘的规定、
向量的叉乘的规定。这四种运算方式的规定本质上仅仅只是对
于向量的特定运算方式的更简洁的更利于研究的表示形式而已。

有了这四种运算方式的规定,我们可以更简洁地将代表两个位移线段
首尾相接后的从末端指向前端的位移线段表示为
$\boldsymbol{a}+\boldsymbol{b}$,将力的大小、路径大小
以及力与路径之间的夹角的乘积表示为$\boldsymbol{F}
\cdot d\boldsymbol{r}$等等。

并且,我们用定义出的新运算符号简洁地表示了本就存在的规律,如
\begin{equation}
    (\boldsymbol{a}+\boldsymbol{b})+\boldsymbol{c}=
    \boldsymbol{a}+(\boldsymbol{b}+\boldsymbol{c}).
\end{equation}
等。且由于时间所限,笔者并没有详细全面地列举和证明向量的规律,
而仅仅只是讲述一种认识向量的思想。
\section{直线运动}
\subsection{机械运动}
我们在初中学过,一个物体相对于另一个物体的位置变化叫做机械运动。
从这里不难看出,机械运动是需要参照系的。尤其需要明确的一点是,
运动总是相对的,而不是绝对的。

\begin{definition}
    被选作参照的另外的物体,被称为参照物。在以后的学习中,为了进行定量
    计算,我们还常常会选定参照物建立参照系。
\end{definition}
\subsection{质点}
\begin{definition}
    当一个物体的大小、形状对结果造成的影响可以忽略时,将该物体看作一个有质量
    而无体积的点来处理,这个等效的点叫做质点。
\end{definition}

在中学当中,质点往往是为了简化问题,这实际上比较容易理解。
在后续的学习中,我们会发现将大的物体分解成无穷个质点来处理是非常
常见的。
\subsection{位置和位移}
\begin{definition}
    在物理学中,我们用位移表示质点位置的变化。位移被定义为向量,
    其大小为始末位置的距离,方向由起始点指向终点。
\end{definition}

值得注意的是,位移与路程是两个不同的物理量。路程是指质点所通过的
实际轨迹的长度,只有大小,没有方向,是个标量。
\subsection{匀速直线运动~速度}
\begin{definition}
    物体在一条直线上运动,如果在相等时间里的位移相等,这种运动就叫做
    匀速直线运动。匀速直线运动有时简称为匀速运动。
\end{definition}

\begin{definition}
    在匀速直线运动中,位移和时间的比值,叫做匀速直线运动的速度。如果 
    做匀速运动的物体在时间$t$内的位移是$s$,速度$v$就可以用下式来表示:
    \begin{equation}
        v=\frac{s}{t}
    \end{equation}
\end{definition}
速度在数值上等于单位时间内位移的大小。在国际单位制中,常常用
米/秒$(m/s)$作为速度的单位。

速度不但有大小,而且有方向,是矢量。速度矢量的方向就是物体位移的
方向。在匀速直线运动中,计算时通常取位移方向作为正方向,速度是正值。

从速度的公式$v=s/t$可以得到 
\begin{equation}
    s=vt
\end{equation}
这个公式叫做匀速运动的位移公式。
\subsection{匀速直线运动的图像}
\subsubsection*{1、位移-时间图像}
任意选择一个平面直角坐标系、用横轴表示时间,用纵轴表示位移,画出
位移和时间的关系图线,这种图象叫做位移-时间图象,简称为位移图象。

显然,位移图像的斜率为
\begin{equation}
    k=\frac{\Delta s}{\Delta t}
\end{equation}
这表示匀速直线运动的速度。
\subsubsection*{2、速度-时间图像}
在平面直角坐标系中,用横轴表示时间,用纵轴表示速度,画出速度和时
间关系的图线,这种图象叫做运动的速度- 时间图象,简称为速度图象。匀速
运动的速度不随时间改变,它的速度图象是一条与横轴平行的直线。

匀速直线运动的速度-时间图像下的面积是
\begin{equation}
    \Delta S=v\Delta t=s
\end{equation}
即面积代表位移的大小。
\subsection{变速直线运动}
\subsubsection*{1、变速直线运动}
\begin{definition}
    物体在一条直线上运动,如果在相等时间里的位移不相等,这种运动就叫
    做变速直线运动。变速直线运动有时简称为变速运动。
\end{definition}

\subsubsection*{2、平均速度}
\begin{definition}
    在变速直线运动中,运动物体的位移和所用时间的比值,叫做这段时间里
的平均速度。如果用 $v$来表示平均速度,那么
\begin{equation}
    \overline{v}=\frac{s}{t}
\end{equation}
\end{definition}

一般认为平均速度是一个标量,其方向意义不大。
\subsubsection*{3、瞬时速度}
\begin{definition}
    运动物体在某一时刻(或某一位置)的速度,叫做顺时速度。
\end{definition}

在数学上,瞬时速度可以被表示为当$\Delta t$趋近于零时,平
均速度的极限值,即 
\begin{equation}
    v=\lim\limits_{\Delta t\to 0}\overline{v}=\lim\limits_{\Delta t\to 0}
    \frac{\Delta s}{\Delta t}
\end{equation}
其中$\lim\limits_{\Delta t\to 0}$表示极限的限制条件。

瞬时速度既有大小,又有方向,是矢量。在直线运动中,瞬时速度的方向就
是物体经过该点时的运动方向。即时速度的大小叫做即时速率,简称速率,它
是一个表示物体运动快慢程度的标量。
\subsection{匀变速直线运动~加速度}
\begin{definition}
    在一条直线上运动的物体,如果在相等的时间里速度的变化相等,物体的
    运动就叫做匀变速直线运动,或者简称为匀变速运动。
\end{definition}

\begin{definition}
    在匀变速直线运动中,速度的变化和所用的时间的比值,叫做匀变速直线
    运动的加速度。
\end{definition}

用$v_0$表示运动物体开始时刻的速度(初速度),用$v_t$表示经过一段时间$t$
的速度(末速度),用$a$表示加速度,那么,
\begin{equation}
    a=\frac{v_t-v_0}{t}
\end{equation}
由上式可以看出,加速度在数值上等于单位时间内速度的变化。

加速度的单位是由时问的单位和速度的单位确定的。在国际单位制中,时
间的单位是秒,速度的单位如果用 $m/s$、加速度的单位就是 $m/s^2$,读作米每二
次方秒。

加速度不但有大小,而且有方向,因此是矢量。

在匀变速直线运动中,加速度矢量是恒定的,大小和方向都不改变,因此
匀变速直线运动也就是加速度矢量恒定的运动。
\subsection{匀变速直线运动的速度}
\subsubsection*{1、匀变速直线运动的速度}
根据定义
\begin{equation}
    a=\frac{v_t-v_0}{t}
\end{equation}
可以得到
\begin{equation}
    v_t=v_0+at
\end{equation}
这个公式叫做匀变速直线运动的速度公式。
\subsubsection*{2、匀变速直线运动的速度图像}
容易知道,匀变速直线运动的速度图像的斜率为 
\begin{equation}
    k=\frac{\Delta v}{\Delta t}=a
\end{equation}
即斜率等同于加速度。

匀变速运动的位移可以用速度图线和横轴之间的面积来表示。可以这样
想,我们把时间细分为很多相等的区段,在每个区段内,物体以区段
开始时的速度进行匀速直线运动,此时这样一个虚拟的运动所经过的
总的位移就是一个个长方形的面积之和。

我们把时间区段分的越小,可以预见这样的找到的虚拟的运动与实际进行的
运动所经过的位移就会越接近,同时,虚拟运动的面积之和也与实际运动的
面积之和越来越接近。随着单个时间区段的减小,虚拟运动的位移会趋近于
实际运动的位移值,同时,虚拟运动的面积也会趋近于实际运动的面积。
而由于虚拟的运动所经过的总的位移就是一个个长方形的面积之和,因此
当我们取的时间间隔无限小时,实际运动的位移就等同于其速度-时间曲线下面的面积。
\subsection{匀变速直线运动的位移}
\subsubsection*{1、匀变速直线运动的位移}
我们已经知道匀变速直线运动的速度公式为
\begin{equation}
    v_t=v_0+at
\end{equation}
而匀变速直线运动的位移可以表示为
\begin{equation}
    s=\sum\lim\limits_{\Delta t\to 0}v_t\Delta t=
    \sum\lim\limits_{\Delta t\to 0}(v_0+at)\Delta t
\end{equation}
并且 
\begin{equation}
    \lim\limits_{\Delta t\to 0}\frac{(t+\Delta t)^2-t^2}{\Delta t}
    =\lim\limits_{\Delta t\to 0}\frac{2t\Delta t+\Delta t^2}{\Delta t}
    =2t
\end{equation}
亦即 
\begin{equation}
    \lim\limits_{\Delta t\to 0}t\Delta t=\frac{1}{2}\Delta (t^2)
\end{equation}
将式(1.63)代入式(1.61),得到 
\begin{equation}
    s=\sum\lim\limits_{\Delta t\to 0}(v_0+at)\Delta t=
    v_0\sum\lim\limits_{\Delta t\to 0}\Delta t+
    \frac{1}{2}a\sum\lim\limits_{\Delta t\to 0}\Delta (t^2)
    =v_0t+\frac{1}{2}at^2
\end{equation}
这就是匀变速直线运动的位移公式。
\subsubsection*{2、两个重要推论}
根据$v_t=v_0+at$,我们有 
\begin{equation}
    t=\frac{v_t-v_0}{a}
\end{equation}
将此式代入匀变速直线运动的位移公式,得到
\begin{equation}
    \begin{aligned}
        s&=v_0\frac{v_t-v_0}{a}+\frac{1}{2}a\left(\frac{v_t-v_0}{a}\right)^2\\
        &=\frac{v_0v_t-v_0^2}{a}+\frac{v_t^2-2v_0v_t+v_0^2}{2a}\\
        &=\frac{2v_0v_t-2v_0^2+v_t^2-2v_0v_t+v_0^2}{2a}\\
        &=\frac{v_t^2-v_0^2}{2a}
    \end{aligned}
\end{equation}
这一结论也常常被表述为
\begin{equation}
    v_t^2-v_0^2=2as
\end{equation}

另外,根据平均速度的定义,我们有 
\begin{equation}
    \overline{v}=\frac{s}{t}=v_0+\frac{1}{2}at=\frac{1}{2}(v_0+v_t)
\end{equation}
该式表明:在匀变速运动中,某段时间内的平均速度等于达段时间的初速
度和末速度的算术平均值,要注意这个结论是利用匀变速运动的公式导出的,
所以它只适用于匀变速运动,对非匀变速运动并不适用。
\subsection{自由落体运动}
\subsubsection*{1、自由落体运动}
\begin{definition}
    物体只在重力作用下,从静止开始下落的运动,叫做自由落体运动。
\end{definition}

\subsubsection*{2、自由落体运动的加速度}
\begin{definition}
    不同的自由落体,它们的运动情况相同:它们在做初速度为零的匀变速运
    动中,在相同的时间内发生相同的位移。由此可以知道,在同一地点,一切物
    体在自由落体运动中的加速度都相同。这个加速度叫做自由落体加速度,也叫
    重力加速度,通常用$ g $来表示。
\end{definition}

目前国际上取$g=9.80665m/s^2$为重力加速度的标准值,在通常的计算中可以取
$g=9.8m/s^2$。匀变速直线运动的公式在自由落体运动中仍是适用的,只不过把
加速度换成了重力加速度$g$。
\subsection{竖直上抛运动}
\begin{definition}
    将物体用一定的初速度沿竖直方向向上抛出去,物体所做的运动叫做竖直
    上抛运动。
\end{definition}

竖直上抛运动和自由落体运动一样,本质上不过是匀变速直线运动规律的应用,
因此我不再赘述详细求解方法。但是值得注意的一点是,正方向的问题。

正方向的问题贯穿中学物理学应用的始终,如果不能理解到正方向的本质,对理解计算的
过程有深远的恶劣影响。以竖直上抛为例,我们稍微介绍下所谓正方向。在计算竖直上
抛运动时,我们可以不考虑分成很多过程,而是仅将其看作一个匀变速直线运动。那么 
我们在计算上很快会发现一个问题:即有时候我们的速度算出来是负值,这实际说明一个
问题:即重力加速度“削减”的速度值大于了初始的速度值,也就是说,实际速度的大小
是此时速度的大小,而方向与初始的上抛方向相反。

对上面这段话做一个总结,就是当我们计算出速度的值为负时,代表速度已经反向。
这就是所谓正方向的本质。因此我们引入了所谓正方向的规定,即我们规定初始速度
的方向为正方向,那么当计算出速度变为负值时,说明其方向与初始速度相反。

理解正方向是一个漫长的过程,读者需要在每次计算中多多留意正方向规定的
合理性,并总是试图寻找其本质,才能真正理解。