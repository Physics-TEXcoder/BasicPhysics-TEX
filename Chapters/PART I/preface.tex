这是一本高中物理教科书,包含力学、热学和电磁学三个部分。
笔者写作这本书,主要是想帮助高中学生能够顺利地理解高中物理的内容。
尽管大家一致认为高中物理的内容是极度幼稚且简单的,但实际上,对于一个
正常的高中生来说,理解高中物理并不是一件简单的事情,这主要是因为通行高中
教科书中并没有讲述原理,例如,完全没有讲述动能定理、机械能守恒的原理,
这在笔者当年读高中时就产生过极大的恶劣影响,以至于长期无法普适论证
机械能守恒的合理性。

并且现在的高中教科书将各部分几乎完全割裂,例如,将静电学、
恒定电流以及磁场完全割裂,以及将力学中的势能和电磁学中的势能以及
其他的如非静电场的势能割裂,使得学生产生物理各部分毫无联系的错觉,以及
莫名其妙的怪异。

为了解决上面所说的问题,笔者写作了这本书。秉持着简约主义的态度,这本书着重
于介绍学生难以在常规的课堂教学和辅导书中学到的物理思想和原理,意图帮助所有
人彻底理解高中物理。

本书的重点在于物理逻辑和思维的传授,并且更注重理论方面的内容,因此仅仅是
对简单的定义和规律等进行了尽可能简洁的说明(这是为了使得本书的内容成为一个
完备的体系,而不是如思考题解答一般零散),并且极多地忽视了应用层面的内容。
并不是笔者在这部分没有技巧和方法,实在是笔者懒得写了,在中学阶段,
应用涉及到的类型可是够多的。而且像定义和规律的详细介绍,以及一些具体的
应用技巧,想必读者在课堂和辅导书上都能学到,因此笔者也没必要在这上面浪费时间。

当然,如果读者有不明白的地方(不管是理论还是应用部分)都可以直接根据封面的
邮箱联系笔者,笔者在空闲时会尽快予以回复。

另外,本书不包含光学和原子物理的内容,因为在中学阶段,这部分内容几乎是
“故事性”的,并没有太多值得深入思考的地方,学生只需自行阅读教材就足够了。
如果后续作者有时间的话,或许会对这两部分进行补充。

关于写作这本书是否有意义的问题,笔者认为是有意义的。尽管这里面的内容
都极其“简单”,以至于可能任何一个物理专业的学生都认为它是幼稚的,但却也总有
大多数人(包括很多物理专业的学生)即使到了大学也没有理解中学物理课程所
涉及的内容。对于以好奇心指引学习的人来说,我们认为奠定一个强大的基础是
重要的,重要程度并不次于学习前沿理论。因此,笔者认为写这本书还是有意义的,
尽管这本书并不会被大范围传播。

希望读者能够从其中获益,并且不吝指教。由于写作时间短暂(实际只用了两三天
时间,但也不必过于担心其中的内容太浅显,因为产生其中内容的过程还算比较漫长,
只是将它们写下来没有费多少时间),并且笔者并不十分了解读者的物理水平,
这本教材里面可能有大量的问题,这可能造成读者阅读的苦难。笔者希望读者能够
将觉得不合理的内容告知笔者,以便笔者改进。当然,即便是写得尽善尽美,这其中的
内容对于中学生来说也不是特别容易理解的,因此读者想要得到收获,必须付出相应的
努力。

并且,本书中存在许多不严谨的内容,主要是由于试图在不超出中学数学
的界限解释物理原理导致的,读者也需要注意这一点。以及,本书在排版、
符号使用等方面也不够标准,甚至存在一些严重问题,比如这一页左上角的
“目录”文字,就是一个排版问题,但笔者暂时不知道如何改掉这一错误,就放任
它留在左上角了。

总而言之,这是一本辅助中学物理学习的教科书,并且不适合脱离其他材料的
辅助单独使用。书写匆忙导致的问题,还望读者谅解。